% Resumo em Português

\vspace{1cm}
\noindent
\par Este relatório de estágio foi realizado no âmbito do estágio inserido no Mestrado em Engenharia Informática-Internet das Coisas da Escola Superior de Tecnologia de Tomar do Instituto Politécnico de Tomar, e tem como objetivo colocar em ambiente real os conhecimentos adquiridos no percurso académico.

\par O estágio tem inerente 4 projetos associados à área do IOT e da monitorização de ambientes e/ou objetos com recurso a diversas soluções. O primeiro projeto, referente ao equipamento "Nidus" já existente. Este equipamento disponibilizado em vários modelos permite a ligação de vários tipos de sensores e atuadores  e centralizar o sistema de monitorização. Este equipamento é capaz igualemente a comunicação dos dados para um portal \textit{Cloud} onde é posível analizar os dados enviados pelos vários equipamentos. Neste projeto é necessário a análise do estado atual do projeto e dar continuidade ao suporte do \textit{Front-end} WEB do equipamento. Este projeto pretende estudar e alterar os métodos de desenvolvimento da página referentes á compressão dos ficheiros para reduzir o espaço ocupado pelas páginas WEB em equipamentos de baixos recursos computacionais e armazenamento, a adoção de um melhor método para utilização de imagens nas interfaces WEB em equipamentos com armazenamento limitado e o desenvolvimento de novas funcionalidades tais como um sistema de internacionalização de modo a disponibilizar o equipamento noutros idiomas, sempre tendo em, conta o baixo armazenamneto disponível. Os restantes projetos serão desenvolvidos de raiz durante o estágio  a pedido dos clientes, por soluções à medida, ou quer pela inovação e evolução dos produtos da empresa, e consistem em novos equipamentos para monitorização.

\par O segundo projeto é referente ao desenvolvimento de um equipamento IOT que tire partido das vantagens da nova tecnologia de comunicação o \textit{Narrowband} (NB-IOT). Neste equipamento irá ser possivel adicionar vários sensores já comercializados pela Captemp. O equipamento é capaz de realizar as leituras de todos os sensores connectados, realizar o armazenamento em Log para posterior envio para o portal \textit{Cloud} para consulta futura ou gestão da alarmística correspondente. Este equipamento é igualmente capaz de realizar uma configuração bi-direcional de modo a ser possível realizar a sua configuração através do portal \textit{Cloud}.

\par Paralelamente aos projetos anteriores e com o emergir de novas técnologias tais como os \textit{Beacons Bluetooth Low Energy} e pela sugestão de diversos clientes de uma nova solução de monitorização portátil e simples, foi criado o projeto "Kea Tracker" composto por \textit{Beacons} e uma aplicação \textit{Mobile} para \textit{Smartphones}, responsável por perioódicamente realizar a leitura dos sensores presentes nos \textit{Beacons}, o seu armazenamento e posterior envio para o portal \textit{Cloud}, à semelhança do projeto anterior. Neste projeto o \textit{Beacon} é igualmente capaz de registar em Log interno para caso exista falha de comunicação com a aplicação \textit{Mobile}, posteriormente esses dados serem descarregados quando o \textit{Smartphone} estiver disponível.

\par O último projeto referente deste estágio, foi solicitado pelo cliente que pretende uma plataforma WEB para localizar pessoas em ambientes \textit{indoor}. Esta solução baseia-se no desenvolvimento de uma solução composta por localizadores, \textit{Gateways} e uma plataforma WEB onde seja possível realizar o \textit{Tracking} de pessoas e de objetos em ambientes \textit{indoor} num mapa e assim gerir os tempos de acesso a zonas definidas, criar alertas, ou simplesmente gerir o stock dos armazéns. Este projeto, à semelhança do projeto "Kea Tracker", usa como tecnologia de suporte \textit{Beacons Bluetooth Low Energy} (BLE) para fazer o tracking em tempo real.




\bigskip

\textbf{Palavras chave:}
IOT, Monitorização, Compressão WEB,Compressão de Imagens,NB-IOT,Beacon,BLE, Indoor Tracking, Bluetooth Tracking