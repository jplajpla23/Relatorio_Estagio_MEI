\par Inserido num contexto empresarial , e tratando-se de projetos em desenvolvimento e de desenvolvimento contínuo, estes não foram concluidos com o concluir deste relatório e estágio. Todos os objetivos defindos no inicio do estágio foram concluidos com sucesso e com bons resultados como apresentado no capítulo \ref{cap4}. 
\par A realização de um estágio com projetos reais em ambiente empresarial permitiu uma melhor compreensão dos vários fatores que  não são possiveis replicar em ambiente académico, além da utilizaão de novas metodologias e técnologias.

\par Para trabalho futuro prevê-se a implementação do plugin Blockly na página da Nidus, algumas alterações simples de Design na página e a reformulação da ferramenta de compressão da página para a utilização do Brotli.
O projeto do Nb-Iot caso exista atualizações de \textit{Firmware} por parte da Digi que permitam a comunicação por Bluetooth, será implementada a configuração do equipamento através de uma aplicação móvel. No projeto dot.Tracker caso a necessidade de aumentar a precisão se venha a verificar, deve-se ponderar a migração para um sistema diferente de localização indoor sem recurso ao valor de RSSI tal como o \textit{Bluetooth} 5.1, em desenvolvimento, especialmente desenhado para localização em ambientes fechados. O \textit{Bluetooth} 5.1 com recurso a várias antenas por recetor, consegue calcular o ângulo entre o emissor e o recetor e com recurso a vários recetores delimitar a posição do emissor.

\par A utilização de \textit{Frameworks} que permitam o desenvolvimento de aplicações multiplataforma, tais como a utilizada neste estágio, o IONIC, é uma mais valia, pois utilizando o mesmo código fonte desenvolvido, permite além de manter o Design similar entre as plataformas, diminui o tempo de desenvolvimento quando são necessárias correções e atualizações e caso não seja necessário pelos requiesitos da aplicação, abstrai o desenvolvimento da programação nativa em cada ambiente, para \textit{Frameworks} de mais alto nível.

\par O estágio na Captemp foi uma experiência enriquecedora, onde foi possivel obter e consolidar conhecimento já adquiridos e novos, experiência e criadas novas relações quer a nível profissional quer pessoal.