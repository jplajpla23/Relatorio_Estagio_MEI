% Abstract in English

\vspace{1cm}
\noindent
\textbf{} 

\par This traineeship report was written in the scope of the traineeship inserted in the master's degree in Computer Engineering -Internet of Things at the School of Technology of Tomar from the Polytechnic Institute of Tomar, and documents the application of the knowledge acquired during this academic degree in a real life environment.


\par This work comprises 4 projects related with the IoT field, regarding environmental and objects monitoring using different solutions. The first project, is related with an existent system named "Nidus". This device, available in several models, allows the connection of several types of sensors and actuators and centralizes the monitoring system. This equipment is equally capable of sending the data to a Cloud portal where it is possible to analyze the data collected by the various devices. In this project it was necessary to analyze the status of the solution used at the time, and give continued support to the WEB Front-end of the equipment. This included the study and reeingineering of the development methods used in the web pages, namely the ones related with file compression in order to reduce the space occupied by those WEB pages, due to the low computational resources and storage of sutch devices. Moreover, several improvements and changes were also done regarding the image resources used by the web interfaces, in order to save storage at the devices. In addition, other features were also implemented, among which the development of a simpliified, lightweight, internationalization system, able to provide support for different languages in the web interfaces while saving precious bytes. 


\par The second project is related with the development of an IOT device that takes advantage of the new communication technology, the Narrowband (NB-IOT). Several contributions were made to this project. Among which, making it possible to add several sensors already sold by Captemp. The equipment is capable of performing the readings of all the connected sensors, making the storage in Log for later sending to the Cloud portal for future preview or management of the corresponding alarms. This equipment is also capable of performing a bi-directional configuration so that it is possible to carry out its configuration through the Cloud portal.


\par In parallel to previous projects and with the emergence of new technologies such as Bluetooth Low Energy Beacons and the suggestion, by several costumers, of a new portable and simple monitoring solution, the "Kea Tracker" project was created. This solution is composed by Beacons and a Mobile application for Smartphones, responsible for periodically reading the sensors present in the Beacons, storing it and sending them to the Cloud portal, at similarity to the previous project. In this project, a Beacon is also able to register in the internal log in case there is a communication failure with the Mobile application, after which these data are downloaded when the Smartphone is available.


\par The last project of the traineeship, consisted of a project requested by a client, who requested a WEB platform to locate people in indor environments. The proposed solution is composed of locators, gateways and a WEB platform where the user is able to perform tracking operations of people and objects in indoor environements on a map. In addition, the user can also manage access times of defined zones, create alerts, or simply manage warehouse stock. This project, like the "Kea Tracker" project, is supported by the BLE technology to do real-time tracking.



\bigskip

\textbf{Key words:} 
IOT, Monitoring, WEB Compression, Image Compression, NB-IOT, Beacon, BLE, Indoor Tracking, Bluetooth Tracking