
\section{Introdução}
Nesta secção é apresentada o estado da arte do projeto realizado durante o estágio na empresa CapTemp. Nessa ordem é apresentado o funcionamento do sistema de configuração do equipamento Nidus desenvolvido pela empresa CapTemp. Na seçcão \ref{Página de configurações do Sistema Nidus} são apresentadas também as metodologias e tecnologias que o sistema implementa de momento para a compressão da página de configurações que dá suporte ao sistema embebido. Na seccção \ref{solucoesDisponiveis} será abordado as soluções existentes na comunidade cientifica que possibilitam a otimização e compressão da página de configurações.

\section{Sistema embebido - Nidus} \label{Sistema embebido - Nidus}
 O sistema Nidus, nas suas diversas versões de hardware disponibiliza ao utilizador uma interface   de configurações
\section{Página de configurações do Sistema Nidus} \label{Página de configurações do Sistema Nidus}
\section{solucoesDisponiveis} \label{solucoesDisponiveis}












\section{1}
sdfdsf
Além de demonstrar a novidade de seus resultados de investigação, um estado da arte tem outras caracteríticas importantes, designadamente:
\begin{itemize}
  \item A leitura da literatura relacionada com o seu problema de investigação e desenvolvimento, contribui decisivamente para a aprendizagem com outros investigadores, tornando mais fácil a análise e compreensão do problema.
  \item Demonstra se o seu problema é relevante. Se muitas pessoas estão a tentar resolver o mesmo problema de investigação e caso o consiga demonstrar no estado da arte, ninguém poderá dizer que o problema em resolução não é importante.
  \item Mostra diferentes abordagens para uma solução. Ao ver muitas abordagens diferentes de outros investigadores, torna possível avaliar a nossa própria abordagem e perceber a sua novidade (ou falta dela) facilmente. Tal também permitirá perceber quais as abordagens mais populares e quais são becos sem saída.
  \item Permite reutilizar o que outros fizeram. Especialmente quando faz investigação sobre novo software, é surpreendente quantas pessoas criaram o software que se pretende desenvolver. Basta fazer uma pesquisa no sourceforge e no github.
\end{itemize}

Então, como escrever um bom estado da arte? Escrever um bom estado da arte depende em 110\% de ter uma definição clara do problema. Se falhou na definição do seu problema com clareza, não conseguirá escrever um bom estado da arte. O motivo é que sem uma definição clara do problema é impossível saber o que pesquisar. Por isso se está a ter problemas no estado da arte revisite a definição do seu problema e se necessário peça ajuda aos seus orientadores! Aqui estão alguns passos / dicas para começar a escrever:

\begin{enumerate}
  \item O estado da arte não é uma via unidirecional. Isto quer dizer que não é numa noite que se escreve o estado da arte. O estado da arte sofre alterações ao longo do trabalho e escrita do relatório. Saber o que outros investigadores estão a fazer deve fazer parte de todo o trabalho de investigação e desenvolvimento que está a realizar. Portanto, um passo importante é criar um sistema de registo e resumo do que vai lendo. Pode para tal usar um software de bibliografia, como por exemplo o Mendeley. É importante que vá registando tudo o que lê por palavras suas.
  \item Seja crítico ao escolher a sua literatura. Não leia tudo. Há muita lixo na web, e não deve perder o seu tempo no lixo. Um critério importante para escolher a sua literatura é garantir que seja \emph{revista por pares} e já tenha sido apresentado ou publicado em conferências ou revistas (com factor de impacto ISI) de renome. No caso de material técnico relacionado com as tecnologias de informação, o IEEE, Elsevier ou a Wiley são bons sítios para começar. Também é uma boa idéia criar uma lista de literatura inicial com os seus orientadores.
  \item Pare de ler! Faça uma seleção inicial de literatura (10-20 documentos, dependendo do problema de pesquisa) e fique com estes por algum tempo. Não continue encontrando novos artigos, ou então nunca terminará a sua tese!
  \item Gaste tempo na análise e não em fazer resumos! Um mero resumo de 10-20 artigos não é um estado da arte. Há software que pode resumir qualquer artigo automaticamente e muito mais rápido do que alguma vez conseguirá. Os seus resumos são um estado da arte somente quando os relaciona com sua própria análise de problemas.
  \item \emph{Dê sempre crédito! Não dar crédito à investigação de outros também é chamado de plágio.}
  \item Para escritores mais avançados: é uma boa prática documentar a sua metodologia para fazer uma revisão bibliográfica. Isso significa que deve documentar como pesquisou a literatura, qual a literatura que incluiu e a que decidiu excluiur, como fez a sua análise e assim por diante. Isto é chamado de revisão sistemática. Em \cite{Kitchenham2007} pode encontrar um guia para fazer revisões sistemáticas na área da engenharia de software.
\end{enumerate}

Sugere-se ainda a leitura do manual sobre revisões bibliográficas da Universidade da Carolina do Norte em \cite{LiteratureReview2017}.
