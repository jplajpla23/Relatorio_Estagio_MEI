\section{Contextualização}
Atualmente a sociedade vive rodeada de tecnologias indispensáveis e de ambientes que se intitulam Smart, mas nem sempre foi assim.\par
Desde cedo, mesmo antes de existir tecnologia, o homem tendeu a procurar e encontrar coisas que melhorassem a sua vida e bem-estar pessoal e da sociedade, mas para chegar a humanidade está hoje é necessário recuar na história algum tempo para marcos importantes da tecnologia.\par
Um dos marcos muito importantes para a descoberta  dos sistemas embebidos foi a invençao dos processadores. Com o surgimento dos processadores começaram a surgir os primeiros sistemas embebidos. Com o passar dos anos até aos dias de hoje a técnologia tem vindo a evoluir e por consequência os sistemas embebidos também se adapataram para os padrões de hoje em dia.\par
Uma das partes mais importantes num sistema embebido é a sua interface disponivel para o utilizador, as principais e mais usadas nos dias de hoje são  a linha de comandos e a WEB, comuns para configuraçoes á distância e as interfaces dos proprios equipamentos como os ecrãs com software proprietário.

{\color{red} \rule{\linewidth}{0.5mm} }
+ JPLA

\section{Motivação e Objetivos}
O estágio é uma forma do estudante colocar numa situação de contexto profissional os conceitos adquiridos em contexto académico. A realização de um estágio é uma mais valia pois possibilita o adquirir de experiência profissional. Ao longo do estágio, serão aplicados vários conhecimentos adquiridos ao longo do percurso académico, tais como técnicas de otimização de código, compressão de ficheiros, manipulamento de imagens,..........\par
{\color{red} \rule{\linewidth}{0.5mm} }

\section{Contribuições}
Nesta secção deve apresentar as contribuições do trabalho, dando especial relevo às que são novidade. Note que a inovação não é obrigatória em trabalhos de mestrado. Portanto nesta secção deve descrever sucintamente todo o trabalho realizado.

\section{Organização da dissertação (opcional)}

{\color{red} \rule{\linewidth}{0.5mm} }

\section{Como utilizar este template}
Esta secção serve para dar algumas instruções sobre a edição de textos em latex e utilização deste template.

