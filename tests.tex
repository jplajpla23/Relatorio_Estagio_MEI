\par Neste capítulo irá ser abordado os testes e a avaliação do trabalho desenvolvido durante o estágio e apresentado no capítulo \ref{workcharp}. À semelhança dos capítulos anteriores, o capítulo está divido em vários um por cada projecto. Todos os projetos obtiveram resultaods positivos em relação aos objetivos abordados no início do estágio.

\section{Nidus}

\par Durante os desenvolvimentos ocuridos durante a realização do estágio foram testados as novas funcionalidades desenvolvidas, nomeadamente em relação à compressão dos ficheiros e das imagens. 
\par A compressão das imagens e utilização dos SVG ao invés dos PNG e JPEG, como foi indicado no subcapítulo \ref{custom}, foi implementada com sucesso e abriu novas possibilidades  no sistema Nidus e pemitiu igualmanete para o seu utilizador uma melhor fluidez nas animações e nas transições.
\par A compressão dos ficheiros com recurso à mudança do método de compressão oringinalmente o GZIP para Brotli obteve bons resultados, como indicado na tabela \ref{tabw2}, nos ficheiros atuais da Nidus e permitiu libertar espaço suficiente para alocar as restantes funcionalidades desenvolvidas no estágio tais como o sistema de internacionalização. ou a adaptcação futura do plugin Blockly na gestão de eventos.

\section{Nb-Iot}

\par Os testes realizados no \textit{firmware} desenvolvido demonstraram a capacidade de este utilizar toda a memória disponível sem ocorrerem necessidade de alocar memória e não ser possível. Comparadamente com o \textit{firmware} inicial, o qual não permitia algumas funcionalidades tais como o envio para o portal das suas configurações e apenas o envio das leituras e com um limite de 25 leituras de Log, o \textit{firmware} desenvolvido possui todas as necessidades do projeto tais como a comunicação bi-direcional entre o portal e o equipamento teve nos testes realizados um limite de 170 Logs com o valor máximo de sensores estípulados(6) o qual pode ser configurado e caso este seja alterado a possibilidades de aumentar a capicadade de Log. Para um máximo de 3 sensores o valor máximo de leituras em Log sobe para 340 e sobe para 1020 no caso de apenas ser alocado o espaço para um sensor.

\section{Kea Tracker}

\par Neste projeto o desenvolvimento limitou-se apenas à aplicação visto durante a realização dos restantes projetos ter sido fornecido uma versão já com suporte às funcionalidades requeridas. Focou-se assim o projeto no desenvolvimento da aplicação multiplataforma a qual integrou todas as funcionalidades já existentes com sucesso e a inclusão das novas funcionalidades  referentes ao \textit{download} dos Logs e ao sincronismo de leituras que permite ao utilizador final uma precisão garantida ao minuto.

\section{dot.tracker}

\par A plataforma dot.Tracker durante os testes demonstrou-se cerca de 5 a 6 vezes mais rápida relativamente ao prótitipo desenvolvido com recurso à \textit{Framework} Laravel, além de corrigir os problemas indentificados e apresentados ao longo deste relatório. Os filtros desenvolvidos/ aplicados conseguem com sucesso atenuar as oscilações presentes nas comunicações, garantindo um movimento mais estável e mais suave. 
\par  A adocção da correção de altura demonstrou ser uma importante fase no cálculo da posição estimada do equipamento. Durante estes testes foi possível constatar, um problema na utilização do método atual para cálculo da distância visto que no caso de ser configurada uma altura na plataforma e o equipamento se mover no eixo Z correspôndente à altura os resultados obtidos aumentam o erro consoante a diferença de altura real e a calculada.  Ainda referente à altura no caso de oscilações muito grandes  e em casos especificos o algoritmo descarta pacotes visto não terem sentido. Supondo o teste realizado com um \textit{beacon} e alguns recetores. Configurando na plataforma a altura do \textit{beacon} a 10 M de altura e a altura do recetor a 0 M de altura, caso seja rececionado com um pacote com o RSSI correspondente a uma altura  perto de 10 M é possivel obter a distãncia a colocar no mapa como elucidado na figura \ref{altura1}. Caso seja movido o \textit{beacon} para uma altura de por exemplo 0.5 M o algoritmo desenvolvido ao compensar a distância com a diferênca de alturas não é capaz de calcular esse valor visto que segundo o algoritmo de Pitágoras não é possivel obter um triângulo rectângulo com um cateto de 10 M e a hipotenusa de 0.5 M. Nestes casos a plataforma não calcula  a sua posição no mapa.
\par A respeito da precisão do sistema em geral é possivel obter uma precisão média entre 1 a 2 metros  consoantes os ambientes, o seu conteúdo ou simplesmente às interferências existentes. De modo a aumentar a  precisão em ambientes mais hostis para localização o aumento do número de recetores e menor distância entre eles permite melhores resultados.