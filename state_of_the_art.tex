
\section{Introdução}
Nesta secção é apresentada o estado da arte dos projetos realizados durante o estágio na empresa CapTemp. Nessa ordem é apresentado o funcionamento do sistema do Nidus desenvolvido pela empresa CapTemp e a sua página de configuração e visualização. Na secção \ref{Página do Coletor de Dados Nidus} são apresentadas também as metodologias e tecnologias que o sistema implementa atualmente para a compressão das páginas que dão suporte ao sistema. Na secção \ref{nbiot} e \ref{kea} irá ser introduzido o plano inicial dos projetos a desenvolver e a base já existente tal como as tecnologias que estes irão utilizar. Na secção \ref{solucoesDisponiveis} será abordado as soluções e tecnologias existentes na comunidade científica e alguns produtos similares, já existentes para os projetos anteriormente referidos.

\section{Coletor de Dados - Nidus} \label{Coletor de Dados - Nidus}
\par
O sistema Nidus, apesar das suas diversas versões de hardware partilha entre todas as versões o mesmo centro de processamento o módulo RCM6760 da Rabbit. O sistema Nidus é composto por dois módulos principais, o Back-end que gere toda a parte de leitura de sensores, de atuação e envio de alertas, log entre as demais funcionalidades e o Front-end, duas páginas WEB Single-Application de modo a não sobrecarregar o módulo com a interface e mover o processamento da interface para o browser do cliente. Na primeira página é possível visualizar os valores obtidos pelo Back-end com atualização em tempo real. Na segunda página e possível carregar as configurações para realizar alterações nas mesmas. A comunicação entre os dois componentes é feita através de XML. Para consultar os valores na primeira página o Front-end acede ao ficheiro values.xml gerado pelo Back-end onde contém todas os valores necessários. Na página de configurações á semelhança da primeira página os valores são carregados por um ficheiro XML o ficheiro setup.xml, incluindo a particularidade de aceitar pedidos POST de modo a alterar as configurações do equipamento.
\par A Nidus dispõe de base para o utilizador variadas funcionalidades tais como, leitura de sensores TH3 e Airo, INPUTS digitais, OUTPUTS digitais e analógico, leitura de sensores SNMP e MODBUS, envio de alertas via GSM e E-mail, programação de eventos, envio automático para um portal Cloud e Log Interno. Outras funcionalidades estão disponíveis mediante o pedido do cliente tais como sensores específicos, leitura de sensores por RS232 ou protocolos de comunicação específicos.
Na tabela \ref{tab0} são apresentadas as principais características do módulo RCM6760 da Rabbit.




\begin{table}[htb]
\centering
\caption{Especificações do Módulo RCM6760}\label{tab0}
\begin{tabular}{|c|c|}\hline
Microprocessor&Rabbit 6000 \\\hline
Frequencia do Microprocessor &200 MHz\\\hline
Flash Memory &4 MB (Código e Sistema de Ficheiros)\\\hline
SRAM&1 MB\\\hline
Power &260 mA 3.3V - Ethernet ON\\\hline
\end{tabular} 
\end{table}

\subsection{Páginas do Coletor de Dados Nidus} \label{Página do Coletor de Dados Nidus}
\par
O código desenvolvido de modo a chegar á fase de produção é comprimido e compilado de modo a que ocupe o mínimo espaço e possa ser armazenado na memória do módulo e coabitar com o Firmware de Back-end, segue os seguintes passos de desenvolvimento:
\begin{enumerate}
\item Desenvolvimento/ alteração do código JavaScript necessário; 
\item Compressão das Imagens necessárias com recurso a ferramentas online tais como o TinyPNG\cite{tinypng} e posterior conversão em Base64 para incluir no JavaScript a imagem e o mesmo poder fazer a gestão da apresentação
\item Compilação/compressão do JavaScript num ficheiro único com recurso ao Google Clousure Platform, nesta etapa para cada versão de hardware é compilado consoante os ficheiros a incluir, poupando o espaço não necessário como o código referente aos Inputs e Outputs na Nidus C, C+ e W, ou o código referente ao módulo wireless nas versões não Wireless.
\item Geração do minificado do código HTML
\item Compressão de cada ficheiro para o seu respetivo GZIP
\end{enumerate}
\par
Após estes passos fica disponível uma nova versão da página pronta a ser carregada na Nidus.
Na imagem \ref{fignidusPage} é apresentado o estado e layout de uma página da Nidus IT no momento do início do estágio.

\begin{figure}[ht]
\centering
\includegraphics[width=0.75\textwidth]{images/layoutPAginaInit.png}
\caption{Layout página da Nidus IT no início do estágio}\label{fignidusPage}
\end{figure}


\section {NB-Iot \& Digi Xbee 3 }\label{nbiot}
\par
Os módulos Xbee 3 representado na figura \ref{figxbee} da DIGI dispõe recentemente de uma versão NB-Iot/ LTE. Ideal para projetos com baixo volume de transmissão de dados e com baixo consumo de energia. O módulo inclui também um compilador de Micropython, contundo a versão Micropython desenvolvida pela DIGI e incluída no módulo XBee, não inclui todas as funcionalidades do Micropyhton tais como por exemplo a biblioteca de gestão de Arrays e o módulo de "\_thread" pois o mesmo não tem suporte para multithread.
Na tabela \ref{tab1} são apresentadas as principais características do módulo XBee 3 da Digi\cite{Digixbee}.

\begin{figure}[ht]
\centering
\includegraphics[width=0.60\textwidth]{images/xbee.jpg}
\caption{Módulo Xbee 3 e placa de expansão desenvolvida pela Captemp}\label{figxbee}
\end{figure}

\begin{table}[htb]
\caption{Especificações do Módulo Xbee 3}\label{tab1}
\begin{tabular}{|c|c|}\hline
Chipset& U-blox SARA-R410M-02B\\\hline
Dimensões& 24.38 mm x 32.94 mm \\\hline 
Temperatura de Funcionamento& -40º C to +85º C \\\hline 
Tipo de SIM & 4FF Nano \\\hline
Interfaces& UART, SPI, USB \\\hline 
Programação MicroPython& 32 KB Flash / 32 KB RAM \\\hline 
I/O& 4 ADC (10-bit), 13 I/O digitais, USB, I2C \\\hline 
Bluetooth& BLE Ready \\\hline 
Potencia de Transmissão& Até 23 dBm \\\hline 
Sensibilidade de Receção (LTE-M) & -105 dBm \\\hline 
Sensibilidade de Receção (NB-IoT) & -113 dBm \\\hline 
Velocidade Downlink/Uplink(LTE-M) & Até 375 kb/s \\\hline 
Velocidade Downlink/Uplink(NB-IoT) & Até 27.2 kb/s Downlink, 62.5kb/s Uplink \\\hline 
Alimentação & 3.3-4.3VDC \\\hline 
Pico corrente na transmissão & \begin{tabular}{@{}c@{}} 550mA - Bluetooth OFF \\ 610mA - Bluetooth ON\end{tabular}\\\hline 
Corrente média de transmissão (LTE-M) & 235mA \\\hline 
Corrente média de transmissão (NB-IoT) & 190mA \\\hline 
Modo Power Save& 20uA \\\hline 
Modo Deep Sleep& 10uA \\\hline 
\end{tabular} 
\end{table}

\par A Captemp pretende, através da utilização deste módulo e de uma placa de expansão desenvolvida pela própria, apresentada anteriormente na figura \ref{figxbee}, desenvolver uma versão do seu outro equipamento de Nb-Iot, mais simples representando numa opção de menor custo para o cliente. Será necessário desenvolver todo o código referente à gestão interna de Logs para guardar informação quando não existe cobertura para envio, o agendamento do envio e leituras, otimização da memória e bateria e implementação de comunicação bidirecional com encriptação com o portal Senslive. Sempre com recurso á programação em MicroPython.
A placa de expansão inclui um módulo de RTC, um conversor 1Wire para possibilitar a leitura de sondas já desenvolvidas pela Captemp, um sistema de alimentação para possibilitar a alimentação por pilha ou por alimentação externa. Ao desenvolver todo o equipamento a empresa tem o controlo total sobre o Firmware e sobre a estrutura de envio e a vantagem de tornar o equipamento compatível com todos os sensores que já possui.

\subsection {MicroPython}
\par O MicroPython\cite{MicroPython}, lançado em 2014, é um compilador e interpretador que implementa a linguagem Python3 e otimiza o seu funcionamento em microcontroladores. Escrito em C e disponibilizado em Open-Source é possível adaptar o mesmo para os diversos equipamentos. \par
É suportado por diversas arquiteturas de processadores tais como:
\par
\begin{itemize}
\item x86
\item x86-64
\item ARM
\item ARM Thumb
\item Xtensa
\end{itemize}
\par
Em microcontroladores que suportem Multi-thread , não sendo o caso do módulo usado está disponível ao programador o módulo de "\_thread" para criar processamento paralelo. Disponibiliza a programação de interrupções físicas, uteis em microcontroladores, tem disponível um "Garbage collector" para gerir a memória do microcontrolador e bibliotecas tais como "usocket" para criação e gestão de sockets, "network" para gerir a comunicação com o módulo específico de cada microcontrolador, ou a biblioteca para gerir o módulo de Bluethooth denominada por "ubluetooth". As bibliotecas disponíveis encontram-se no Site oficial da documentação\cite{micropython_lib}. 

\subsection {NB-Iot/ LTE-M}
O NB-Iot ou Narrowband Iot  e o LTE-M são tecnologias de Low Power Wide Area. São indicadas para sistemas Smart em diversas áreas como a monotorização, a agricultura, localizadores entre outras áreas. Similar ao funcionamento da rede móvel, onde cada equipamento possui um cartão SIM e se liga á rede fornecida pelo operador, mas utilizado em equipamentos com menor transmissão de dados e que não tem acesso a fontes de alimentação fixas e requerem de baterias, o NB-Iot promete autonomias das baterias a rondar os 10 anos\cite{u_2017}.Devido ao baixo volume de dados o plano de dados é possível apenas com pequeno investimento obter anos e até décadas de transmissões da dados.
\par De entre as vantagens podem-se destacar:
\begin{itemize}
\item Baixo Consumo
\item Longo alcance e boa penetração
\item Baixo custo de desenvolvimento na implementação da cobertura
\item Custo reduzido pelas transmissões
\item Sem necessidade de Roaming
\end{itemize}
\par
A cobertura da rede está a ser implementada pelas operadoras de telecomunicações que já possuem cobertura da rede GSM e infraestrutura de ligação á rede Internet desenvolvida e apenas necessitam de 
disponibilizar cobertura nas antenas de rede móvel, normalmente já existe compatibilidade de Hardware e basta atualizações de Firmware. É aconselhado pelas operadoras que se utilize o Nb-Iot para equipamentos fixos e o LTE-M para equipamentos em movimento.

\subsubsection { Low Power Wide Area}
As redes Low Power Wide Area são redes usadas frequentemente no IOT quando é necessário enviar dados a distâncias longas. Combinam a largura de banda e o consumo de bateria presente em redes como BLE e Zigbee, com alcance igual ou superior ás redes de comunicação GSM. São caracterizadas por ter longo alcance, um baixo custo de transmissão e baixo consumo, onde simples baterias podem fornecer alimentação na ordem das décadas. Este alcance pode ser conseguido por exemplo por redes multihop ou modulações especificas que privilegiem o consumo energético e o alcance. A comunicação 2G e 3G pode ser usada em comunicação M2M mas as mesmas tem uma largura de banda superior ao necessário o que resulta em consumo de bateria excessivo onde não é tirado proveito da largura de banda disponível. Alguns exemplos de redes Low Power Wide Area, ou simplesmente denominadas por LPWAN, são o DASH7, o SigFox, LoRa, Ingenu, Telensa ou o NarrowBand Iot.\cite{lpwanoverview}

\begin{figure}[ht]
\centering
\includegraphics[width=0.45\textwidth]{images/lpwan.png}
\caption{Gráfico com relação Distancia vs Largura de Banda\cite{masterthesisLPWAN}}\label{figgraphlpwan}
\end{figure}



\section {Kea Tracker}\label{kea}
O Projeto Kea Tracker utiliza Beacon’s da Ruuvi, uma Beacon open-source\cite{ruuvi}, que disponibiliza de forma open-source tanto o Firmware para alterações, como as aplicações para Android e IOS. Será desenvolvida uma aplicação baseada na aplicação fornecida e o Firmware para disponibilizar a funcionalidade de data-logger.
\subsection{Beacons BLE}
\par
O Bluetooth Low Energy ou simplesmente BLE foi desenvolvido a pensar nos novos equipamentos IOT, onde os utilizadores querem vários equipamentos ligado ao mesmo tempo. Para tal foi desenvolvido o BLE que permite mais ligações ao mesmo tempo comparando com o Bluetooth clássico.
Como é indicado no nome, o principal fator diferenciador nesta versão, utilizada muitas vezes em equipamentos IOT, é o baixo consumo de aproximadamente metade relativamente ao Bluetooth normal. Outras características melhoradas a visar os equipamentos de IOT no BLE são a baixa largura de banda e o baixo tempo de transmissão.

Com o desenvolver do BLE foram criados novos tipos de equipamentos, nomeadamente as beacons, equipamentos quase sempre alimentados por pilhas, que comunicam através de BLE, tornando o equipamento portátil. As beacons são caracterizadas por transmitir pequenas quantidades de informação em Broadcasting.
Existem dois tipos de beacons as beacons não conectáveis e as conectáveis\cite{blepacket}. Como indicado no nome as beacons conectáveis permitem que um equipamento (como um smartphone) se conecte á beacons e esta fica preparada para receber dados. As não conectáveis apenas permitem o broadcasting dos dados, poupando energia pois apenas é necessário ter o módulo acordado para fazer o broadcast e o restante do tempo podem estar num estado sleep. Na figura \ref{blepacket} é apresentado o pacote que é transmitido em broadcast para os outros equipamentos ao alcance.

\begin{figure}[htb]
\centering
\includegraphics[width=0.65\textwidth]{images/blepacket.png}
\caption{BLE Broadcast packet\cite{blepacket}}\label{blepacket}
\end{figure}


\subsection{Ruuvi Beacons}
\par Neste projeto o firmware das beacons necessita de uma alteração, tornar a beacon numa beacon conectável e esta armazenar internamente as ultimas leituras num buffer circular e criar um data-logger e caso o cliente pretenda poderá conectar mais tarde para fazer o download para aplicação e posterior envio para o Senslive, não necessitando a proximidade do smartphone á beacon durante todo o tempo. A Ruuvi dispõe de dois modos de desenvolvimento de firmware da beacon em C ou usando o Espruino, á semelhança do MicroPython um interpretador de JavaScript para microcontroladores lançado em 2012, totalmente compatível com as beacons da Ruuvi.
\subsection{Apps Smartphones}
Na fase inicial será adaptada a versão disponibilizada para Android para agilizar a integração com o portal Senslive. A aplicação base para android disponibilizada pela Ruuvi foi desenvolvida em Kotlin\cite{ruuviappgithub}, uma linguagem desenvolvida pela JetBrains multiplataforma e que inclui o Android nessas plataformas compatíveis.
De seguida estão apresentadas algumas alterações necessárias na aplicação:
\begin{itemize}
\item Alteração das Imagens e Logotipo da App;
\item Alteração do Nome da App;
\item Remoção de conteúdo não necessário;
\item Bloqueio do URL de envio para usar exclusivamente o portal Senslive;
\item Melhoramento da precisão da posição GPS;
\item Possibilidade da alteração dos intervalos de registo
\end{itemize}


\section{Soluções e Tecnologias Disponíveis} \label{solucoesDisponiveis}
\subsection{Tecnologias Disponíveis}
\subsubsection{Compressão de Ficheiros}
\par
Atualmente a vida online do Homem passou a ter um grande impacto na sua vida. Para tal as páginas web e seus conteúdos foram aumentado em quantidade e tamanho e com menores tempos de resposta. Isso é aplicável tanto aos ficheiros que contem o layout da página, quer das imagens. Para poupar dados de transmissão e reduzir tempos de envios, ou simplesmente suportar larguras de banda inferiores, os browsers integraram a possibilidade de receber os ficheiros comprimidos e fazer a descompressão para mostrar ao cliente quase em tempo real. Atualmente os browser recentes suportam a compressão por GZIP( já utilizado na página do equipamento Nidus) e compressão utilizado a codificação Brotlin \cite{Alakuijala2019} \cite{brotlirfc}.
Cada método de compressão posui as suas vantagens e desvantagens, o brotli por sua vez á semelhança de outros métodos comparadamente ao GZIP, tem uma taxa de compressão superior\cite{Alakuijala2015}, isto significa que consegue reduzir o mesmo ficheiro no seu respetivo ficheiro comprimido ocupando menos espaço comparadamente com o GZIP, mas como desvantagem o tempo de compressão do mesmo é superior. O tempo e velocidade de descompressão é superior no Brotli do que nas restantes alternativas.
\par
O GZIP e o brotli usam na sua compressão para reduzir o tamanho do ficheiro o algoritmo de compressão LZ77 , que procura sequências repetidas utilizando o método de janela deslizante e substitui essas sequências por referências para a primeira ocorrência que não foi substituída indicando a distancia a que a primeira ocurencia ocorre e o tamanho a substituit. Na figura \ref{gzip} e \ref{gzip2} é apresentado dois exemplos visuais e simples utilizando letras e frases de como o LZ277, uzado pelo GZIP e Brotli, comprime os ficheiros. Na Figura \ref{unzip} é apresentado o ficheiro base, neste caso representado por um pequeno texto. No exemplo apresentado pela figura \ref{gzip} apenas foi utilizado a substituição de palavras inteiras, na figura \ref{gzip2} procura sequências de caracteres sejam elas palavras ou não.
\begin{figure}[htb]
\centering
\includegraphics[width=0.95\textwidth]{images/FILE.png}
\caption{Sequência não comprimida}\label{unzip}
\end{figure}

\begin{figure}[htb]
\centering
\includegraphics[width=0.95\textwidth]{images/gzip.png}
\caption{Sequência comprimida com LZ77 (apenas palavras)}\label{gzip}
\end{figure}
\begin{figure}[htb]
\centering
\includegraphics[width=0.95\textwidth]{images/gzip2.png}
\caption{Sequência comprimida com LZ77(palavras e sequências)}\label{gzip2}
\end{figure}

\subsubsection{Compressão de Imagens}
\par

O utilizador pretende igualmente ver as imagens com a máxima qualidade, mais qualidade significa um maior detalhe e por sequencia um ficheiro de maior tamanho. Existem atualmente vários softwares online e locais que reduzem o tamanho das imagens. Na conceção da página da Nidus é utilizado o website TinyPNG.com que analisa a imagem original e converte as cores em cores mais simples de o sistema armazenar, como por exemplo uma imagem com 24 bits de profundidade de cor pode ser convertido em uma similar com apenas 8 bits reduzindo o tamanho do ficheiro e impercetível para o olho humano num ecrã\cite{Hilles2019}. Alternativamente ao Tiny Png existem softwares, similares alguns de licença GNU/GPL, para comprimir imagens. O "Mass Image Compressor"\cite{Mass_Image_Compressor}(apenas um exemplo), é possível comprimir as imagens com a possibilidade de indicar a quantidade de compressão.
\par
Com a enorme quantidade e diversidade de monitores existentes, as páginas web necessitam de ser responsivas e apresentar a melhor imagem para o monitor em questão, isso normalmente traduz-se em várias versões similares da imagem alojadas no servidor. No caso dos microcontroladores e sistemas embebidos o espaço encontra-se limitado e deve-se arranjar uma solução. Uma solução possível é ao invés da utilização de imagens PNG, JPG ou outras, é a utilização de imagens em SVG, onde a imagem é representada por um ficheiro XML que descreve uma imagem bidimensional e utiliza na sua constituição modelos matemáticos para o cálculo das posições dos elementos. Com isto é possível manipular o XML em tempo real para alterar elementos ou remover, alterar cores, criar animações entre outras. Inclui a vantagem de como a imagem é representada por formulas matemáticas, é possível escalar a imagem sem perder qualidade pois a função matemática é ajustável. Num sistema embebido como o caso da Nidus é vantajoso a utilização de imagens em SVG para criação das animações. Atualmente as animações da página da Nidus são criadas com várias imagens PNG comprimidas e convertidas em base64 e são alternadas no HTML pelo JavaScript. Com a utilização de imagens SVG é possível ter apenas uma imagem alojada e manipular a imagem em tempo real através do JavaScript de uma forma mais suave para o utilizador, pois apenas a zona a alterar é alterada na imagem.
Á semelhança dos JPG e PNG o SVG também pode ser comprimido, para tal basta no XML da definição remover os meta-dados  e utilização de funções mateméticas mais simples, não necessários para o browser apresentar a reprentação gráfica do mesmo, mas os softwares de edição adicionam para funcionalidades esclusivas do editor. Á semelhança dos ficheiros HTML após a remoção dos meta-dados o ficheiro pode ser minificado.

\subsection{Produtos Similares}
\subsubsection{NB-Iot}
\par
Atualmente no mercado começam a surgir alguns produtos similares ao que se pretende desenvolver como é o caso dos sensores da Efento\cite{epoka}, que disponibiliza vários tipos de sensores que comunicam por NB-Iot. A Efento é uma empresa fundada em 2014 e é focada em desenvolvimento de equipamentos IOT. Atualmente desenvolveram versões com suporte para NB-Iot. Estes equipamentos tem a desvantagem de não ser compatível com o pacote de envio desenvolvido no portal Senslive e apenas permite o envio para o portal da Efento e não existe a possibilidade da utilização das sondas já comercializadas pela Captemp. Como vantagem á semelhança do equipamento a desenvolver é a utilização de um sistema com Log para quando não existe possibilidade de comunicação.
Devido ao desenvolvimento da tecnologia ainda existem poucas soluções em comercialização, estando as mesmas em desenvolvimento. A Captemp possui igualmente outro equipamento, completamente desenvolvido pela empresa, em desenvolvimento que tira partido do NB-Iot com o acréscimo em relação ao que se pretende desenvolver durante o estágio, a possibilidade de ter mais sensores, maior capacidade de Log interno, configuração por Bluetooth, GPS e um Display integrado como extra.
\subsubsection{Kea Tracker}
Após pesquisas online é possível encontrar algumas soluções de beacons que permitem o armazenamento interno de leituras para desenvolver um sistema de data-logger tais como a Beacon da Fujitsu, a FWM8BLZ02A-109069\cite{beacon1} , á semelhança da beacon da Ruuvi usa o mesmo chip o nRF52832 da Nordic Semiconductor, mas apresenta como vantagens a inclusão de um sistema de Logs interno com capacidade para aproximadamente 4080 leituras e a diversidade de sensores já incluídos. Como desvantagem em relação á Beacon da Ruuvi tem a inclusão de um sensor de temperatura ao invés de temperatura e humidade, não possui sensor de pressão atmosférica e não é open-source possuindo um firmware fechado.
\par
Outra solução existente no mercado é igualmente a solução da Blue Maestro que possui variadas versões de beacons. Á semelhança da Beacon da Fujitsu possuem igualmente sistema de Log. Contrariamente á FWM8BLZ02A-109069 é uma beacon que tem disponível em Open-Source uma API e um SDK para desenvolver as nossas aplicações. Comparada com a beacon da Ruuvi, a Ruuvi beacon é completamente open-source e não apenas a API para comunicação.
\par
Na tabela \ref{tabbeacons} são apresentadas as diferenças e semelhanças entre os três modelos analisados

\begin{table}[htb]
\caption{Comparação entre beacons \cite{specsrect}\cite{bluespecs}\cite{ruuvispecs}}\label{tabbeacons}
\begin{tabular}{|c|c|c|c|}\hline
& Ruuvi Tag& Fujitsu Beacon &Blue Maestro \\\hline
Processador& nRF52832& nRF52832 &? \\\hline
Memória&\begin{tabular}{@{}c@{}}512kB Flash \\ 64kB RAM\end{tabular} & 32K Não volátil &?\\\hline 
Protocolos&\begin{tabular}{@{}c@{}c@{}@{}c@{}} Bluetooth 5 \\ Wirepass \\ Mira OS\\QUUPA\\Others (2.4GHz)\end{tabular}&Bluetooth 4.1&BLE 4.2\\\hline 
\begin{tabular}{@{}c@{}}Potência de\\ Transmissão\end{tabular} &+4 dBm &\begin{tabular}{@{}c@{}}-16, -12, -8\\ -4, 0, +4 dBm\end{tabular} &-4, 0, +4 dBm \\\hline
Sensores& \begin{tabular}{@{}c@{}c@{}c@{}} Acelerometro\\ Temperatura\\ Humidade \\Pressão\end{tabular} &\begin{tabular}{c@{}c@{}} Acelerómetro\\ Temperatura\end{tabular}&\begin{tabular}{@{}c@{}c@{}} Temperatura\\ Humidade \\Pressão\end{tabular}\\\hline 
NFC & \checkmark&- &-\\\hline
Bateria &\begin{tabular}{@{}c@{}}CR2477\\ 1000mAH - Li/MnO2\end{tabular}&CR2450 &CR2032\\\hline
\begin{tabular}{@{}c@{}}Autonomia\\(espetável)\end{tabular}& ~10 Anos&1 Ano em Broadcast &\begin{tabular}{@{}c@{}}1 Ano em Broadcast\\2 Anos com Log\end{tabular}\\\hline
Data Logger &\begin{tabular}{@{}c@{}}-\\(a desenvolver)\end{tabular}&\checkmark &\checkmark\\\hline
Open Source & \checkmark&- &\checkmark ( API \& SDK )\\\hline
Informações & \begin{tabular}{@{}c@{}c@{}@{}} IP67 \\ 2 Botões\\2 Leds\\52mm \diameter\\\end{tabular}&\begin{tabular}{c@{}c@{}} Led\\ 40 x 31 x 12mm \end{tabular}&\begin{tabular}{@{}c@{}}24000 Registos\\33mm \diameter\end{tabular} \\\hline
\end{tabular} 
\end{table}
\par
