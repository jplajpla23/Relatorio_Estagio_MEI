% Preamble
\documentclass[a4paper, twoside, 12pt]{report}

% % -------------------- New packages  ----------------------------
% \usepackage{multirow}
%\usepackage{changepage}
%\usepackage{titlesec}
\usepackage{amsmath}
\usepackage{float}
\usepackage{fancyhdr}

\usepackage[linesnumbered,ruled,vlined]{algorithm2e}
\usepackage[noend]{algpseudocode}
\algnewcommand\algorithmicforeach{\textbf{for each}}
\algdef{S}[FOR]{ForEach}[1]{\algorithmicforeach\ #1\ \algorithmicdo}

\usepackage{array}
\newcolumntype{P}[1]{>{\centering\arraybackslash}p{#1}}

\setcounter{secnumdepth}{3}

\usepackage{textcomp}
\usepackage{gensymb}


\usepackage{pgfplots}
\usepackage{pgfplotstable} 


% % ------------------------------------------------

% Includes
		% UTF-8 encoding, so that yogoou can use characters like ç and ã

\usepackage[T1]{fontenc}				% Same, but for output encoding
\usepackage[utf8]{inputenc}		
\usepackage{tikz}
\def\checkmark{\tikz\fill[scale=0.4](0,.35) -- (.25,0) -- (1,.7) -- (.25,.15) -- cycle;} 
\def\diameter{\tikz\filldraw[fill=white!60,line width=0.15mm](0,0) circle[radius=0.15cm] -- (-0.18,-0.18) -- (-0.19,-0.19) -- (-0.19,-0.19)-- (0.18,0.18) -- cycle;} 

\usepackage[portuges]{babel}	% Still related to the above
\usepackage{acronym}					% List of acronyms
\usepackage{textcomp} 					% Extra characters
\usepackage{graphicx} 					% \includegraphics{}, the most common command to include images in figures
\usepackage{titlesec}					% To manually format the chapter titles
\usepackage[left=3cm,right=2.5cm,top=2.5cm,bottom=2.5cm]{geometry} % Margins, as dictated by the rules
%\usepackage[nottoc,numbib]{tocbibind} 	% Hyperlinks in table of contents, useful for navigation
\usepackage[section]{placeins}			% \FloatBarrier, a useful command when your figures are trying to run away
\usepackage{caption}					% For captioning figures
\usepackage{subcaption}					% Subfigures (the subfigure package is deprecated and should not be used)
\usepackage[toc,page]{appendix}			% Appendices
\usepackage{pdfpages}					% Useful when your appendix is a pre-compiled PDF, such as a whole paper
\usepackage{url}						% Useful when one wants to include URLs in the text
\usepackage[
      colorlinks=true,    			%no frame around URL
      urlcolor=black,    			%no colors
      menucolor=black,    			%no colors
      linkcolor=black,    			%no colors
      citecolor=black,    			%no colors
      bookmarks=true,    			%tree-like TOC
      bookmarksopen=true,    		%expanded when starting
      bookmarksnumbered=true, 		%Put section numbers in bookmarks
      hyperfootnotes=true,    		%no referencing of footnotes, does not compile
      pdfpagemode=UseOutlines,    	%show the bookmarks when starting the pdf viewer
      plainpages=false, 			%solve problem ``destination with the same identifier'' warning
      pdfpagelabels				 	%solve problem ``destination with the same identifier'' warning
]{hyperref} 							% So that our citations look good and still work as links
\usepackage{epigraph}					% For your inspirational quote
\usepackage{etoolbox}
\usepackage{enumitem}
\usepackage{listings,xcolor}
\usepackage{algpseudocode, algorithm2e}
\usepackage{algcompatible}
\usepackage{booktabs}
\usepackage{multirow}
\usepackage{emptypage}
%\usepackage{subfigure}
%\usepackage{fontspec}

\setlength{\parindent}{2em}

\setlength{\headheight}{16pt}
\renewcommand{\baselinestretch}{1.3}	% 1.5 line spacing, as mandated by the rules
%\titleformat{\chapter}[hang] 			% Smaller chapter titles
%{\normalfont\huge\bfseries}{\thechapter}{1em}{}

% Your info goes here
\newcommand{\thesistitle}{Relatório de Estágio MEI-IdC}			% Your work's title
\newcommand{\myname}{João Paulo Lopes Agostinho}				% Your name
\newcommand{\statedate}{Tomar, Março 2020}					% The date, usually "Place, Month Year"
\newcommand{\supervisorname}{Renato Eduardo da Silva Panda, Instituto Politécnico de Tomar}		% Your supervisor's name
%\newcommand{\cosupervisorname}{Professora Doutora Marie Curie}	% Your co-supervisor's name, if any.


\renewcommand\lstlistingname{}
\renewcommand\lstlistlistingname{Algorithms} %!!!!!!!!!!!!!!!!!!!!!!!!!!!!!!!!!!!!!!!!!!!!!!!!!!!!!!!!!!!!!!!!!!!!!!!!!!!!!!!!!!

%\DeclareUnicodeCharacter{00A0}{~}

\makeatletter
\renewcommand*{\cleardoublepage}{\clearpage\if@twoside \ifodd\c@page\else
\hbox{}%
\thispagestyle{empty}%
\newpage%
\if@twocolumn\hbox{}\newpage\fi\fi\fi}
\makeatother

\pagestyle{plain}
\addto\captionsportuges{
  \renewcommand{\contentsname}%
    {ÍNDICE}%
}
\addto\captionsportuges{
  \renewcommand{\listtablename}%
    {ÍNDICE DE TABELAS}%
}
\addto\captionsportuges{
  \renewcommand{\listfigurename}{ÍNDICE DE FIGURAS}
}

\renewcommand\appendixtocname{Apêndice}
\renewcommand\appendixpagename{Apêndice}


% MAIN DOCUMENT
\begin{document}
\pagestyle{headings}
\pagenumbering{roman}

\includepdf[pages={-}]{images/cover/capa.pdf}
% Blank page
\newpage
\thispagestyle{empty}
\mbox{}
% Title page 1
%\begin{titlepage}
\thispagestyle{empty}

\begin{center}
% IPT Logo and Name
\includegraphics[width=0.8\textwidth]{images/logo_ipt.jpg}
% Thesis name
\vspace{1cm}
{\huge{\textbf{\thesistitle}}\par}
\vspace{1.5cm}
{{\large{Dissertação de Mestrado}}\par}

\vspace{1.5cm}
{\large{\textbf{Orientado por:}\\
Prof. Dr. Einstein \\
Prof. Dr. Marie Curie \\
Prof. Dr. Darth Vader
}}

\vspace{1cm}
{\large{\textbf{Juri:} \\
Prof. Dr. Steve Jobs \\
Prof. Dr. Bill Gates \\
Prof. Dr. Mark Zuckerberg
}}

% Final Stuff
\vfill
Dissertação apresentada ao Instituto Politécnico de Tomar para cumprimento dos requisitos necessários à obtenção do grau de Mestre em Engenharia Informática – Internet das Coisas

\vspace{0.5cm}
{\large \statedate\par}


\end{center}
\end{titlepage}  
% Blank page
%\newpage
%\thispagestyle{empty}
%\mbox{}

% Agradecimentos
\titleformat{\chapter}[display]{\rmfamily\Large\bfseries}{\thechapter}{0.5ex}{\centering}[\vspace{-0.5ex}\rule{\textwidth}{0.3pt}]
\chapter*{AGRADECIMENTOS}
\addcontentsline{toc}{chapter}{Agradecimentos}
\titleformat
{\chapter} % command
[display] % shape
{\bfseries\Large\itshape} % format
{Story No. \ \thechapter} % label
{0.5ex} % sep
{
    \rule{\textwidth}{1pt}
    \vspace{1ex}
    \centering
} % before-code

% You can add blank pages here, if you like
\newpage\null\thispagestyle{empty}\newpage

% RESUMO
\titleformat{\chapter}[display]{\rmfamily\Large\bfseries}{\thechapter}{0.5ex}{\centering}[\vspace{-0.5ex}\rule{\textwidth}{0.3pt}]
\chapter*{RESUMO}
\addcontentsline{toc}{chapter}{Resumo}
% Resumo em Português

\vspace{1cm}
\noindent
\par Este relatório de estágio foi realizado no âmbito do estágio inserido no Mestrado em Engenharia Informática-Internet das Coisas da Escola Superior de Tecnologia de Tomar do Instituto Politécnico de Tomar, e tem como objetivo colocar em ambiente real os conhecimentos adquiridos no percurso académico.
\par
O estágio tem inerente 4 projetos associados à área do IOT e da monitorização de ambientes e/ou objetos com recurso a diversas soluções. Um dos projetos, "Nidus" já existia e é necessário a análise do projeto para dar continuidade ao suporte do \textit{Front-end}. Neste projeto é necessário estudar e alterar os métodos de desenvolvimento da página referentes á compressão dos ficheiros para reduzir o espaço ocupado pelas páginas WEB, a adoção de um melhor método para utilização de imagens nas interfaces WEB em equipamentos com baixos recursos e o desenvolvimento de novas funcionalidades tais como um sistema de internalização de modo a disponibilizar o equipamento noutros idiomas. Os restantes projetos serão desenvolvidos de raiz durante o estágio quer por pedido de soluções à medida pelos clientes quer pela inovação/evolução dos produtos da empresa, e consistem em novos equipamentos para monitorização, nomeadamente monitorização ambiental e \textit{Tracking} de objetos.
\par O segundo projeto é referente ao desenvolvimento do de um equipamento IOT que tira partido da nova tecnologia de comunicação o \textit{NB-IOT}. Neste equipamento é possivel adicionar vários sensores, e o equipamento é capaz de realizar as leituras dos diversos sensores, o armazenamento em Log para posterior envio para um portal \textit{Cloud} para consulta futura ou criação de alertas.
\par Paralelamente ao projeto anterior e com o emergir de novas técnologias tais como as \textit{Beacons Bluetooth Low Energy} e pela requesição por diversos clientes de uma nova solução de monitorização portátil e simples, foi criado o projeto "Kea Tracker" composto com várias \textit{Beacons} e uma aplicação \textit{Mobile} responsável por ler os sensores presentes nas \textit{Beacons} e o envio para o portal \textit{Cloud}. Neste projeto a \textit{Beacon} deve igualmente possuir um sistema de Log interno para a falha de comunicação com a aplicação \textit{Mobile}.
\par O último projeto do estágio, solicitado pelo cliente, baseia-se no desenvolvimento de uma solução composta por Sensores e \textit{Gateways} e uma plataforma WEB que seja possível realizar o \textit{Tracking} de pessoas e de objetos em ambientes \textit{indoor} e assim gerir tempos de acesso criar alertas, ou simplesmente gerir o stock de armazéns. Este projeto usa como técnologia de suporte \textit{Beacons Bluetooth Low Energy} (BLE) para fazer o tracking em tempo real.




\bigskip

\textbf{Palavras chave:}
IOT, Monitorização, Compressão WEB,Compressão de Imagens,NB-IOT,Beacon,BLE, Indoor Tracking, Bluetooth Tracking

\newpage\null\thispagestyle{empty}\newpage 

%ABSTRACT
\titleformat{\chapter}[display]{\rmfamily\Large\bfseries}{\thechapter}{0.5ex}{\centering}[\vspace{-0.5ex}\rule{\textwidth}{0.3pt}]
\chapter*{ABSTRACT}
\addcontentsline{toc}{chapter}{Abstract}
% Abstract in English

\vspace{1cm}
\noindent
\textbf{} 


\bigskip

\textbf{Key words:} 
% And here as well
\newpage\null\thispagestyle{empty}\newpage

% INSPIRATIONAL QUOTE
% Setup
\setlength\epigraphwidth{12cm}
\setlength\epigraphrule{0pt}
\makeatletter
\patchcmd{\epigraph}{\@epitext{#1}}{\itshape\@epitext{#1}}{}{}
\makeatother
% Actual Quote
\vspace*{\fill}
\epigraph{"Persistence is the shortest path to success"}{}
{ ---  \textup{Charles Chaplin}}
\vspace*{\fill}
\newpage\null\thispagestyle{empty}\newpage

% TABLE OF CONTENTS

\titleformat{\chapter}[display]{\rmfamily\Large\bfseries}{\thechapter}{0.5ex}{\centering}[\vspace{-0.5ex}\rule{\textwidth}{0.3pt}]
\tableofcontents
\clearpage
% LIST OF ACRONYMS
\chapter*{ACRÓNIMOS}
\addcontentsline{toc}{chapter}{Acrónimos}
\begin{acronym}[PROJECT\_NAME]

%\newacronym{BLE}{BLE}{Bluetooth Low Energy}
%\newacronym{GSM}{GSM}{Global System for Mobile Communications}
%\newacronym{JPG}{JPG}{Joint Photographic Group}
%\newacronym{HTML}{HTML}{HyperText Markup Language
%\newacronym{NB-IOT}{NB-IOT}{Narrowband IoT}
%\newacronym{PNG}{PNG}{Portable Network Graphics}
%\newacronym{RTC}{RTC}{Real Time Clock}
%\newacronym{SVG}{SVG}{Scalable Vector Graphics}



\end{acronym}

\addcontentsline{toc}{chapter}{Índice de Figuras}
\listoffigures
\clearpage %\cleardoublepage %for openright
\addcontentsline{toc}{chapter}{Índice de Tabelas}
\listoftables
\clearpage %\cleardoublepage %for openright
% BODY
\newpage
\thispagestyle{empty}
\mbox{}

\fancyhead[LE,RE]{\slshape\rightmark}
\fancyhead[LO,RO]{\slshape\leftmark}
\fancyhead[RE,LO]{}
\pagestyle{fancy}
\titleformat{\chapter}[display]
    {\normalfont\huge\bfseries}{\chaptertitlename\ \thechapter}{20pt}{\Huge}
\titlespacing*{\chapter}{0pt}{0pt}{20pt}

\chapter{Introdução}
\pagenumbering{arabic}
\section{Contextualização}
\par
Atualmente a sociedade vive rodeada de tecnologias indispensáveis e de ambientes que se intitulam Smart, mas nem sempre foi assim.\par
Desde cedo, mesmo antes de existir tecnologia, o homem tendeu a procurar e encontrar coisas que melhorassem a sua vida e bem-estar pessoal e da sociedade, mas para chegar a humanidade está hoje é necessário recuar na história algum tempo para marcos importantes da tecnologia.\par
Um dos marcos muito importantes para o desenvolvimento dos sistemas embebidos e de sistemas de monotorização foi a invenção dos processadores. Com o surgimento dos processadores começaram a surgir os primeiros sistemas embebidos e sistemas de monotorização. Com o passar dos anos até aos dias de hoje a tecnologia tem vindo a evoluir e por consequência os sistemas também se adaptaram para os padrões de atualmente.\par
Uma das partes mais importantes num sistema embebido é a sua interface disponível para o utilizador, as principais e mais usadas nos dias de hoje são a linha de comandos e a WEB, comuns para configurações á distância e as interfaces dos próprios equipamentos como os ecrãs com software proprietário.

\section{A Empresa}
\par
A empresa CapTemp, Lda localizada em Pombal, Leiria é uma empresa, focada em desenvolvimento de soluções de monotorização, controlo, supervisão e de soluções á medida consoante os requisitos do cliente. Para criar um sistema de monotorização é necessário o sistema possuir sensores, atuadores, coletores de dados e software para analisar os dados provenientes dos sensores de modo a possuir capacidade de atuar com base nesses valores. A Captemp é responsável pelo desenvolvimento de todos estes componentes passando pelos sensores até ao software responsável por analisar e armazenar os dados.\par
Uma das subáreas da empresa é a disponibilização de um Registador de temperatura e respetivo Software certificado para Meteorologia Legal.
Faz parte deste conjunto o Software “CapTemp SQL” representado na figura \ref{figcaptempsql} responsável por guardar os dados provenientes dos sensores ligados ao registador.\par
\begin{figure}[ht]
  \centering
  \includegraphics[width=1.00\textwidth]{images/captemp.png}
  \caption{CapTemp SQL}\label{figcaptempsql}
\end{figure}
O registador desenvolvido pela Captemp, representado na figura \ref{fignidusCl} denomina-se por Nidus-C, um registador que suporta até 32 sensores.\par

\begin{figure}[ht]
  \centering
  \includegraphics[width=0.45\textwidth]{images/nidus.jpg}
  \caption{Coletor de Dados Nidus-C}\label{fignidusCl}
\end{figure}

Com a necessidade de mais funcionalidades, a Captemp criou várias variantes da Nidus-C, representadas na Figura \ref{fignidusall} para aplicar em outras áreas para além da Meteorologia Legal. Das quais surgiram a Nidus-C+, similar á Nidus-C acrescentando a possibilidade de adicionar sensores Wireless. A Nidus-IT e Nidus-IT+ duas versões com as funcionalidades da Nidus-C e Nidus-C+ respetivamente, acrescentando Inputs e Outputs ao sistema de monotorização. Para soluções exclusivamente Wireless nasce a Nidus-W suportando apenas sensores Wireless. Por último é desenvolvido a Nidus-R, baseada na Nidus-IT especialmente desenhada a pensar em ambientes IT com suporte para montagem em bastidores.
\par
\begin{figure}[ht]
  \centering
  \includegraphics[width=0.65\textwidth]{images/nidusall.png}
  \caption{Universo Nidus}\label{fignidusall}
\end{figure}

No setor dos sensores foi desenvolvido o TH3 um conversor RS485 permitindo às diversas Nidus, ligar por RS485 a sensores 1Wire além dos dois inputs possuídos no TH3. Nos sensores wireless, foi desenvolvido o Airo á semelhança do TH3 possui dois inputs, um ecrã e possibilita a ligação de sensores. Permite ainda a leitura de todos os Airo adicionados á Nidus ao mesmo tempo, tecnologia desenvolvida pela Captemp denominada por Captemp AST \cite{Captemp_AST}. Ambos os sensores estão representados na Figura \ref{figairoth3} 
\begin{figure}[ht]
  \centering
  \includegraphics[width=0.45\textwidth]{images/th3airo.png}
  \caption{ TH3 e Airo}\label{figairoth3}
\end{figure}
\par
Em desenvolvimento encontram-se sensores com recurso a tecnologias NB-Iot, Beacon's BLE e Lora.
\par
A Captemp desenvolve igualmente um portal Cloud denominado Senslive(Figura \ref{figsenslive}) que possibilita a centralização dos sistemas de monotorização numa plataforma Cloud.

\begin{figure}[ht]
  \centering
  \includegraphics[width=0.95\textwidth]{images/mwsnap0791.png}
  \caption{ Portal Senslive}\label{figsenslive}
\end{figure}



\section{Motivação e Objetivos}
\par
O estágio é uma forma do estudante colocar numa situação de contexto profissional os conceitos adquiridos em contexto académico. A realização de um estágio é também uma mais valia pois possibilita o adquirir de experiência profissional que não é possível obter em contexto escolar.
\par
Ao longo do estágio, serão aplicados vários conhecimentos adquiridos durante o percurso académico de modo a melhorar a interface para o utilizador, técnicas de otimização de código, compressão de ficheiros, manipulação de imagens de modo a ocupar o mínimo de espaço permitindo futuros desenvolvimento e melhorias, dando continuidade ao suporte do projeto Nidus, igualmente serão criados dois novos projetos de desenvolvimento de novos equipamentos que tiram partido de novas tecnologias como o NB-Iot e Beacon’s BLE.
\subsection{Nidus}
\par
O projeto "Nidus" tem como objetivo dar suporte ao Front-end das Nidus já existentes para correções de bugs encontrados em versões anteriores, otimização de código, de modo a ocupar o mínimo espaço, possibilitando deixar memória livre para desenvolvimentos futuros, desenvolver versões customizadas com layouts a pedido do cliente com funcionalidades especificas, ou simplesmente melhorar a página seguindo a tendência de equipamentos concorrentes.
\subsection{NB-Iot}
Com o surgimento da nova tecnologia NB-Iot surgiu a necessidade de serem criados equipamentos que tirem partido dessa tecnologia e as suas vantagens. Para tal durante o estágio será desenvolvido um dos equipamentos que tira partido da tecnologia. Este projeto tem como por objetivo criar uma versão de raiz, simplificada e mais barata de um outro equipamento de NB-Iot em desenvolvimento pela Captemp, através do módulo Xbee da DIGI e da sua programação em Micropython. Durante o projeto será necessário garantir a correta gestão de memória, gestão de Logs internos, comunicação com os sensores físicos, comunicação bidirecional e encriptação com o portal Senslive.
\subsection{Kea Tracker}
O "Kea Tracker" é um projeto de Beacon’s BLE que comunicam com o smartphone, onde é possível definir alertas locais no smartphone e envio dos dados obtidos dos sensores das beacon’s e envio para a plataforma Senslive.
Tal como o projeto anterior será necessário além de criar uma aplicação para smartphone, criar Firmware específico para as beacon’s que na ausência de comunicação com o smartphone devem armazenar em Log as leituras dos sensores e quando este está ao alcance descarregar para o smartphone.
\subsection{dot.Tracker}
A pedido de um cliente foi solicitado o desenvolvimento de uma plataforma para localização de pessoas e objetos em ambientes interiores. O cliente pretende ter uma plataforma onde seja capaz de ver em tempo real a posição de pessoas e objetos definidos previamente, definir zonas de alerta, e consultar o histórico de movimentos. Neste projeto iram ser usadas beacon's BLE e um gateway BLE que dispõe de vários recetores BLE colocados estratégicamente no edificio e responsáveis por receber os broadcasts das beacons que por sua vez transmitem para o gateway que implementa um cliente de MQTT onde são disponibilizadas as mensagens recebidas pelos recetores BLE. O projeto é constituído pelo desenvolvimento da plataforma de gestão e visualização, pelo recetor dos pacotes provenientes do Broker MQTT e respetivos cálculos segundo o algoritmo a adotar.

\section{Os Problemas}
A página WEB da nidus desde a sua criação já sofreu muitas alterações para seguir os padrões e tendências da concorrência e, portanto, está em constante atualização. Hoje em dia com a mundialização quase todas as pessoas sabem inglês e usam sistemas em inglês, mas existem algumas pessoas que ou não sabem ou preferem usar a língua nativa preferem ter o sistema na sua língua, para tal a captemp pretende desenvolver uma página WEB com um sistema de tradução que seja possível alojar na memória do equipamento para o utilizador escolher a linguagem a usar e assim cativar mais clientes a usar os equipamentos Captemp e expandir a Captemp para outros países. Com o acréscimo do sistema de tradução surge o problema de uma quantidade maior de código alojado na memória, irá ser revisto otimizações que se possam fazer no código já existente, irá ser estudado o melhor método de compressão da página mantendo o GZIP utilizado atualmente ou migrar para outro mais recente como o Brotli e compressão de imagens migrando as imagens existentes para imagens SVG, possibilitando outras soluções para a página com sistemas mais interativos e ocupando o menor espaço disponível. Além dos problemas referidos anteriormente poderão surgir novas funcionalidades a implementar, a pedido do cliente, como por exemplo páginas com layout especificos ou novos sensores, ou a simples correção de possíveis Bugs encontrados nas versões em produção
\par
Outro problema a resolver detetado pelo feed-back recebido dos clientes é a complexidade para a criação de eventos, ações e reações, que controlam o Sistema Nidus. Para isso a Captemp pretende reformular a estrutura de gestão de eventos para um sistema mais visual  e atual similar ao Scratch, um software utilizado atualmente para ensinar a crianças as bases da programação e elas mesmos criarem alguns programas sem saber nenhuma linguagem de programação. Na figura \ref{scratch} é apresentado um exemplo de programação usando a ferramenta Scratch, onde o utilizador com um sistema de blocos pode criar condiçoes e eventos a despoletar consoante algumas condições.
\begin{figure}[ht]
  \centering
  \includegraphics[width=0.50\textwidth]{images/scratch.png}
  \caption{ Programação com a ferramenta Scratch}\label{scratch}
\end{figure}
\begin{figure}[htb]
  \centering
  \includegraphics[width=0.20\textwidth]{images/ds1921.jpg}
  \caption{Data Logger iButton}\label{ds1921}
\end{figure}
Outros problemas existentes, a resolver durante o estágio, são a criação de sistemas low-cost, de outros equipamentos CapTemp, para o cliente que não necessita de tantas funcionalidades com a introdução da alternativa para NB-Iot com recurso ao módulo Xbee da Digi, e a substituição de produtos antigos descontinuados, os data-logger(Figura \ref{ds1921}) e sua substituição por similares com as mesmas funções e mais tipos de sensores disponíveis, uma necessidade também já requisitada pelos clientes que pretendem monitorizar mais grandezas além da temperatura, mas com os padrões e tecnologias dos dias de hoje e com suporte para o novo Portal da Captemp o Senslive. Ou simplesmente o desenvolvimento de novos produtos a pedido dos clientes.

\par
Em resumo os problemas a solucionar durante o estágio podem ser encontrados na seguinte lista:
\begin{itemize}
\item Melhorar a compressão da página WEB da Nidus;
\item Melhorar a compressão das imagens presentes na página WEB da Nidus;
\item Correção de Bugs da página Web da Nidus;
\item Melhorar o processo de criação de eventos;
\item Criação de uma página com sistema de tradução automático;
\item Versões customizadas da página WEB a pedido do cliente;
\item Seguir as tendências da concorrencia;
\item Criação de soluções/equipamentos de baixo custo;
\item Substituição de produtos descontinuados;
\item Desenvolvimento  de produtos á medida do cliente.
\end{itemize}

\section{Organização do relatório}

\par Este presente relatório está dividido em 5 capítulos em que o primeiro capítulo onde é feita uma breve introdução ao tema e é apresentado os objetivos, o enquadramento do estágio e alguns aspetos inicias a considerar. 
\par No capítulo seguinte é apresentado a tecnologia e hardware pesquisado com fim a dar suporte a este mesmo estágio e uma pequena pesquisa sobre projetos/produtos similares quer na finalidade quer nas tecnologias usadas. 
\par O capítulo 3 apresenta o trabalho desenvolvido durante o estágio na empresa. É neste capítulo que é apresentado as soluções escolhidas. 
\par No capítulo 4 é  apresentado os testes efetuados baseados na implementação do capítulo 4, demonstrando o funcionamento do trabalho desenvolvido e a avaliação dos mesmos. 
\par Por fim no capítulo 5 é apresentada uma breve conclusão de todo o trabalho, dificuldades e algumas sugestões para futuras implementações. 


	% Arabic numbering starts

% For each chapter, you should have a bit of code that looks like this:
% \label allows you to later \ref that chapter.
% \input includes a different .tex file, so that you can have you dissertation
% neatly partitioned into several files. I recommend one file per chapter.
\chapter{Estado da Arte}
\label{chap:state_of_the_art}

\section{Introdução}
Nesta secção é apresentada o estado da arte dos projetos realizados durante o estágio na empresa CapTemp. Nessa ordem é apresentado o funcionamento do sistema do Nidus desenvolvido pela empresa CapTemp e a sua página de configuração e visualização. Na secção \ref{Página do Coletor de Dados Nidus} são apresentadas também as metodologias e tecnologias que o sistema implementa atualmente para a compressão das páginas que dão suporte ao sistema. Na secção \ref{nbiot}, \ref{kea} e  \ref{dot} irá ser introduzido o plano inicial dos projetos a desenvolver e a base já existente tal como as tecnologias que estes irão utilizar. Na secção \ref{solucoesDisponiveis} será abordado as soluções e tecnologias existentes na comunidade científica e alguns produtos similares, já existentes para os projetos anteriormente referidos.

\section{Coletor de Dados - Nidus} \label{Coletor de Dados - Nidus}
\par
O sistema Nidus, apesar das suas diversas versões de hardware partilha entre todas as versões o mesmo centro de processamento o módulo RCM6760 da Rabbit. O sistema Nidus é composto por dois módulos principais, o Back-end que gere toda a parte de leitura de sensores, de atuação e envio de alertas, log entre as demais funcionalidades e o Front-end, duas páginas WEB Single-Application de modo a não sobrecarregar o módulo com a interface e mover o processamento da interface para o browser do cliente. Na primeira página é possível visualizar os valores obtidos pelo Back-end com atualização em tempo real. Na segunda página e possível carregar as configurações para realizar alterações nas mesmas. A comunicação entre os dois componentes é feita através de XML. Para consultar os valores na primeira página o Front-end acede ao ficheiro values.xml gerado pelo Back-end onde contém todas os valores necessários. Na página de configurações á semelhança da primeira página os valores são carregados por um ficheiro XML o ficheiro setup.xml, incluindo a particularidade de aceitar pedidos POST de modo a alterar as configurações do equipamento.
\par A Nidus dispõe de base para o utilizador variadas funcionalidades tais como, leitura de sensores TH3 e Airo, INPUTS digitais, OUTPUTS digitais e analógico, leitura de sensores SNMP e MODBUS, envio de alertas via GSM e E-mail, programação de eventos, envio automático para um portal Cloud e Log Interno. Outras funcionalidades estão disponíveis mediante o pedido do cliente tais como sensores específicos, leitura de sensores por RS232 ou protocolos de comunicação específicos.
Na tabela \ref{tab0} são apresentadas as principais características do módulo RCM6760 da Rabbit.




\begin{table}[htb]
\centering
\caption{Especificações do Módulo RCM6760}\label{tab0}
\begin{tabular}{|c|c|}\hline
Microprocessor&Rabbit 6000 \\\hline
Frequencia do Microprocessor &200 MHz\\\hline
Flash Memory &4 MB (Código e Sistema de Ficheiros)\\\hline
SRAM&1 MB\\\hline
Power &260 mA 3.3V - Ethernet ON\\\hline
\end{tabular} 
\end{table}

\subsection{Páginas do Coletor de Dados Nidus} \label{Página do Coletor de Dados Nidus}
\par
O código desenvolvido de modo a chegar á fase de produção é comprimido e compilado de modo a que ocupe o mínimo espaço e possa ser armazenado na memória do módulo e coabitar com o Firmware de Back-end, segue os seguintes passos de desenvolvimento:
\begin{enumerate}
\item Desenvolvimento/ alteração do código JavaScript necessário; 
\item Compressão das Imagens necessárias com recurso a ferramentas online tais como o TinyPNG\cite{tinypng} e posterior conversão em Base64 para incluir no JavaScript a imagem e o mesmo poder fazer a gestão da apresentação
\item Compilação/compressão do JavaScript num ficheiro único com recurso ao Google Clousure Platform, nesta etapa para cada versão de hardware é compilado consoante os ficheiros a incluir, poupando o espaço não necessário como o código referente aos Inputs e Outputs na Nidus C, C+ e W, ou o código referente ao módulo wireless nas versões não Wireless.
\item Geração do minificado do código HTML
\item Compressão de cada ficheiro para o seu respetivo GZIP
\end{enumerate}
\par
Após estes passos fica disponível uma nova versão da página pronta a ser carregada na Nidus.
Na imagem \ref{fignidusPage} é apresentado o estado e layout de uma página da Nidus IT no momento do início do estágio.

\begin{figure}[ht]
\centering
\includegraphics[width=0.75\textwidth]{images/layoutPAginaInit.png}
\caption{Layout página da Nidus IT no início do estágio}\label{fignidusPage}
\end{figure}


\section {NB-Iot \& Digi Xbee 3 }\label{nbiot}
\par
Os módulos Xbee 3 representado na figura \ref{figxbee} da DIGI dispõe recentemente de uma versão NB-Iot/ LTE. Ideal para projetos com baixo volume de transmissão de dados e com baixo consumo de energia. O módulo inclui também um compilador de Micropython, contundo a versão Micropython desenvolvida pela DIGI e incluída no módulo XBee, não inclui todas as funcionalidades do Micropyhton tais como por exemplo a biblioteca de gestão de Arrays e o módulo de "\_thread" pois o mesmo não tem suporte para multithread.
Na tabela \ref{tab1} são apresentadas as principais características do módulo XBee 3 da Digi\cite{Digixbee}.

\begin{figure}[ht]
\centering
\includegraphics[width=0.60\textwidth]{images/xbee.jpg}
\caption{Módulo Xbee 3 e placa de expansão desenvolvida pela Captemp}\label{figxbee}
\end{figure}

\begin{table}[htb]
\caption{Especificações do Módulo Xbee 3}\label{tab1}
\begin{tabular}{|c|c|}\hline
Chipset& U-blox SARA-R410M-02B\\\hline
Dimensões& 24.38 mm x 32.94 mm \\\hline 
Temperatura de Funcionamento& -40º C to +85º C \\\hline 
Tipo de SIM & 4FF Nano \\\hline
Interfaces& UART, SPI, USB \\\hline 
Programação MicroPython& 32 KB Flash / 32 KB RAM \\\hline 
I/O& 4 ADC (10-bit), 13 I/O digitais, USB, I2C \\\hline 
Bluetooth& BLE Ready \\\hline 
Potencia de Transmissão& Até 23 dBm \\\hline 
Sensibilidade de Receção (LTE-M) & -105 dBm \\\hline 
Sensibilidade de Receção (NB-IoT) & -113 dBm \\\hline 
Velocidade Downlink/Uplink(LTE-M) & Até 375 kb/s \\\hline 
Velocidade Downlink/Uplink(NB-IoT) & Até 27.2 kb/s Downlink, 62.5kb/s Uplink \\\hline 
Alimentação & 3.3-4.3VDC \\\hline 
Pico corrente na transmissão & \begin{tabular}{@{}c@{}} 550mA - Bluetooth OFF \\ 610mA - Bluetooth ON\end{tabular}\\\hline 
Corrente média de transmissão (LTE-M) & 235mA \\\hline 
Corrente média de transmissão (NB-IoT) & 190mA \\\hline 
Modo Power Save& 20uA \\\hline 
Modo Deep Sleep& 10uA \\\hline 
\end{tabular} 
\end{table}

\par A Captemp pretende, através da utilização deste módulo e de uma placa de expansão desenvolvida pela própria, apresentada anteriormente na figura \ref{figxbee}, desenvolver uma versão do seu outro equipamento de Nb-Iot, mais simples representando numa opção de menor custo para o cliente. Será necessário desenvolver todo o código referente à gestão interna de Logs para guardar informação quando não existe cobertura para envio, o agendamento do envio e leituras, otimização da memória e bateria e implementação de comunicação bidirecional com encriptação com o portal Senslive. Sempre com recurso á programação em MicroPython.
A placa de expansão inclui um módulo de RTC, um conversor 1Wire para possibilitar a leitura de sondas já desenvolvidas pela Captemp, um sistema de alimentação para possibilitar a alimentação por pilha ou por alimentação externa. Ao desenvolver todo o equipamento a empresa tem o controlo total sobre o Firmware e sobre a estrutura de envio e a vantagem de tornar o equipamento compatível com todos os sensores que já possui.

\subsection {MicroPython}
\par O MicroPython\cite{MicroPython}, lançado em 2014, é um compilador e interpretador que implementa a linguagem Python3 e otimiza o seu funcionamento em microcontroladores. Escrito em C e disponibilizado em Open-Source é possível adaptar o mesmo para os diversos equipamentos. \par
É suportado por diversas arquiteturas de processadores tais como:
\par
\begin{itemize}
\item x86
\item x86-64
\item ARM
\item ARM Thumb
\item Xtensa
\end{itemize}
\par
Em microcontroladores que suportem Multi-thread , não sendo o caso do módulo usado está disponível ao programador o módulo de "\_thread" para criar processamento paralelo. Disponibiliza a programação de interrupções físicas, uteis em microcontroladores, tem disponível um "Garbage collector" para gerir a memória do microcontrolador e bibliotecas tais como "usocket" para criação e gestão de sockets, "network" para gerir a comunicação com o módulo específico de cada microcontrolador, ou a biblioteca para gerir o módulo de Bluethooth denominada por "ubluetooth". As bibliotecas disponíveis encontram-se no Site oficial da documentação\cite{micropython_lib}. 

\subsection {NB-Iot/ LTE-M}
O NB-Iot ou Narrowband Iot  e o LTE-M são tecnologias de Low Power Wide Area. São indicadas para sistemas Smart em diversas áreas como a monotorização, a agricultura, localizadores entre outras áreas. Similar ao funcionamento da rede móvel, onde cada equipamento possui um cartão SIM e se liga á rede fornecida pelo operador, mas utilizado em equipamentos com menor transmissão de dados e que não tem acesso a fontes de alimentação fixas e requerem de baterias, o NB-Iot promete autonomias das baterias a rondar os 10 anos\cite{u_2017}.Devido ao baixo volume de dados o plano de dados é possível apenas com pequeno investimento obter anos e até décadas de transmissões da dados.
\par De entre as vantagens podem-se destacar:
\begin{itemize}
\item Baixo Consumo
\item Longo alcance e boa penetração
\item Baixo custo de desenvolvimento na implementação da cobertura
\item Custo reduzido pelas transmissões
\item Sem necessidade de Roaming
\end{itemize}
\par
A cobertura da rede está a ser implementada pelas operadoras de telecomunicações que já possuem cobertura da rede GSM e infraestrutura de ligação á rede Internet desenvolvida e apenas necessitam de 
disponibilizar cobertura nas antenas de rede móvel, normalmente já existe compatibilidade de Hardware e basta atualizações de Firmware. É aconselhado pelas operadoras que se utilize o Nb-Iot para equipamentos fixos e o LTE-M para equipamentos em movimento.

\subsubsection { Low Power Wide Area}
As redes Low Power Wide Area são redes usadas frequentemente no IOT quando é necessário enviar dados a distâncias longas. Combinam a largura de banda e o consumo de bateria presente em redes como BLE e Zigbee, com alcance igual ou superior às redes de comunicação GSM. São caracterizadas por ter longo alcance, um baixo custo de transmissão e baixo consumo, onde simples baterias podem fornecer alimentação na ordem das décadas. Este alcance pode ser conseguido por exemplo por redes multihop ou modulações especificas que privilegiem o consumo energético e o alcance. A comunicação 2G e 3G pode ser usada em comunicação M2M mas as mesmas tem uma largura de banda superior ao necessário o que resulta em consumo de bateria excessivo onde não é tirado proveito da largura de banda disponível. Alguns exemplos de redes Low Power Wide Area, ou simplesmente denominadas por LPWAN, são o DASH7, o SigFox, LoRa, Ingenu, Telensa ou o NarrowBand Iot.\cite{lpwanoverview}

\begin{figure}[ht]
\centering
\includegraphics[width=0.45\textwidth]{images/lpwan.png}
\caption{Gráfico com relação Distancia vs Largura de Banda\cite{masterthesisLPWAN}}\label{figgraphlpwan}
\end{figure}



\section {Kea Tracker}\label{kea}
O Projeto Kea Tracker utiliza Beacon’s da Ruuvi, uma Beacon open-source\cite{ruuvi}, que disponibiliza de forma open-source tanto o Firmware para alterações, como as aplicações para Android e IOS. Será desenvolvida uma aplicação baseada na aplicação fornecida e o Firmware para disponibilizar a funcionalidade de data-logger.
\subsection{Beacons BLE}
\par
O Bluetooth Low Energy ou simplesmente BLE foi desenvolvido a pensar nos novos equipamentos IOT, onde os utilizadores querem vários equipamentos ligado ao mesmo tempo. Para tal foi desenvolvido o BLE que permite mais ligações ao mesmo tempo comparando com o Bluetooth clássico.
Como é indicado no nome, o principal fator diferenciador nesta versão, utilizada muitas vezes em equipamentos IOT, é o baixo consumo de aproximadamente metade relativamente ao Bluetooth normal. Outras características melhoradas a visar os equipamentos de IOT no BLE são a baixa largura de banda e o baixo tempo de transmissão.

Com o desenvolver do BLE foram criados novos tipos de equipamentos, nomeadamente as beacons, equipamentos quase sempre alimentados por pilhas, que comunicam através de BLE, tornando o equipamento portátil. As beacons são caracterizadas por transmitir pequenas quantidades de informação em Broadcasting.
Existem dois tipos de beacons as beacons não conectáveis e as conectáveis\cite{blepacket}. Como indicado no nome as beacons conectáveis permitem que um equipamento (como um smartphone) se conecte á beacons e esta fica preparada para receber dados. As não conectáveis apenas permitem o broadcasting dos dados, poupando energia pois apenas é necessário ter o módulo acordado para fazer o broadcast e o restante do tempo podem estar num estado sleep. Na figura \ref{blepacket} é apresentado o pacote que é transmitido em broadcast para os outros equipamentos ao alcance.

\begin{figure}[htb]
\centering
\includegraphics[width=0.65\textwidth]{images/blepacket.png}
\caption{BLE Broadcast packet\cite{blepacket}}\label{blepacket}
\end{figure}


\subsection{Ruuvi Beacons}
\par Neste projeto o firmware das beacons necessita de uma alteração, tornar a beacon numa beacon conectável e esta armazenar internamente as ultimas leituras num buffer circular e criar um data-logger e caso o cliente pretenda poderá conectar mais tarde para fazer o download para aplicação e posterior envio para o Senslive, não necessitando a proximidade do smartphone á beacon durante todo o tempo. A Ruuvi dispõe de dois modos de desenvolvimento de firmware da beacon em C ou usando o Espruino, á semelhança do MicroPython um interpretador de JavaScript para microcontroladores lançado em 2012, totalmente compatível com as beacons da Ruuvi.
\subsection{Apps Smartphones}
Na fase inicial será adaptada a versão disponibilizada para Android para agilizar a integração com o portal Senslive. A aplicação base para android disponibilizada pela Ruuvi foi desenvolvida em Kotlin\cite{ruuviappgithub}, uma linguagem desenvolvida pela JetBrains multiplataforma e que inclui o Android nessas plataformas compatíveis.
De seguida estão apresentadas algumas alterações necessárias na aplicação:
\begin{itemize}
\item Alteração das Imagens e Logotipo da App;
\item Alteração do Nome da App;
\item Remoção de conteúdo não necessário;
\item Bloqueio do URL de envio para usar exclusivamente o portal Senslive;
\item Melhoramento da precisão da posição GPS;
\item Possibilidade da alteração dos intervalos de registo
\end{itemize}

\section {dot.Tracker}\label{dot}
Á semelhança do projeto Kea Tracker o projeto dot.Tracker usa igualmente beacon's BLE para enviar a informação necessária para o respetivo portal. É necessário recolher os pacotes recebidos das beacons envia-los para o servidor e calcular a distância entre a beacon e o receptor e com o auxilio de multiplos recetores realizar a triangulação da beacon num mapa. No decorrer do projeto será necessário desenvolver uma plataforma web para receber e visualizar as localizações proveninentes das beacons e respetivas configurações, adotar o método de algoritmo para a triangulação da beacon relativamente a vários recetores e realizar testes ao funcionamento e precisão do sistema.
\subsection{Beacons e gateway}
 Para este projeto irá ser utilizado durante o desenvolvimento a solução da Beacon Line\cite{taskit} e posteriormente desenvolvido recetores propriétarios da Captemp. A solução apresentada pela Beacon Line, é composta por um gateway e vários nós. Cada nó possui um recetor BLE e quando o mesmo recebe um broadcast proveniente da beacon o transmite para o gateway. Caso exista alguma divergência da potência de transmissao desde o ultimo pacote enviado por essa mesma beacon o gateway com connectividade ethernet realiza o publish num broker onde é possivel o servidor obter os pacotes das beacon's.



\section{Soluções e Tecnologias Disponíveis} \label{solucoesDisponiveis}
\subsection{Tecnologias Disponíveis}
\subsubsection{Compressão de Ficheiros}
\par
Atualmente a vida online do Homem passou a ter um grande impacto na sua vida. Para tal as páginas web e seus conteúdos foram aumentado em quantidade e tamanho e com menores tempos de resposta. Isso é aplicável tanto aos ficheiros que contem o layout da página, quer das imagens. Para poupar dados de transmissão e reduzir tempos de envios, ou simplesmente suportar larguras de banda inferiores, os browsers integraram a possibilidade de receber os ficheiros comprimidos e fazer a descompressão para mostrar ao cliente quase em tempo real. Atualmente os browser recentes suportam a compressão por GZIP( já utilizado na página do equipamento Nidus) e compressão utilizado a codificação Brotlin \cite{Alakuijala2019} \cite{brotlirfc}.
Cada método de compressão possui as suas vantagens e desvantagens, o brotli por sua vez á semelhança de outros métodos em comparação com o GZIP, tem uma taxa de compressão superior\cite{Alakuijala2015}, isto significa que consegue reduzir o mesmo ficheiro no seu respetivo ficheiro comprimido ocupando menos espaço em relação ao GZIP, mas como desvantagem o tempo de compressão do mesmo é superior. Ao contrário da compressão, na descompressão o Brotli tem melhores resultados do que nas restantes alternativas apresentando velocidades superiores de descompressão.
\par
O GZIP e o brotli usam na sua compressão para reduzir o tamanho do ficheiro o algoritmo de compressão LZ77, que procura sequências repetidas utilizando o método de janela deslizante e substitui essas sequências por referências para a primeira ocorrência que não foi substituída indicando a distancia a que a primeira ocurencia ocorre e o tamanho a substitui.
\par O sistema de janela deslizante define um tamanho da janela e ao deslocar a janela do tamanho definido define um dicionário. Após definir o dicionário com vários tamanhos de janelas, percorrer novamente o ficheiro através do método de janela deslizante novamente a procurar repetições das entradas que existem no dicionário. Quando uma sequência é encontrada esta é substituida por uma referencia da posição da primeira ocurrência da mesma. Na figura \ref{janela} é apresentado um exemplo do funcionamento da janela deslizante para a obtenção do dicionário com o tamanho da janela a variar de 2 a 7.
\begin{figure}[htb]
\centering
\includegraphics[width=0.85\textwidth]{images/janeladeslizantedicionario.png}
\caption{Funcionamento da Janela Deslizante}\label{janela}
\end{figure}


\par Na figura \ref{gzip} e \ref{gzip2} é apresentado dois exemplos visuais e simples utilizando frases de como o LZ277, usado pelo GZIP e Brotl através do sistema de janela deslizante procura as repetições e comprime os ficheiros. Na Figura \ref{unzip} é apresentado o ficheiro base, representado por um pequeno texto. No exemplo apresentado pela figura \ref{gzip} apenas foi utilizado a substituição de palavras inteiras, na figura \ref{gzip2} procura sequências de caracteres sejam elas palavras ou não. Nos exemplos apresentados a redução foi de 20\% [$1-\dfrac{48}{60}\times100\%$] no primeiro exemplo e de aproximadamente de 32\% [$1-\dfrac{41}{60}\times100\%$] no segundo.
\begin{figure}[htb]
\centering
\includegraphics[width=0.85\textwidth]{images/FILE.png}
\caption{Sequência não comprimida}\label{unzip}
\end{figure}

\begin{figure}[htb]
\centering
\includegraphics[width=0.85\textwidth]{images/gzip.png}
\caption{Sequência comprimida com LZ77 (apenas palavras)}\label{gzip}
\end{figure}
\begin{figure}[htb]
\centering
\includegraphics[width=0.85\textwidth]{images/gzip2.png}
\caption{Sequência comprimida com LZ77(palavras e sequências)}\label{gzip2}
\end{figure}


\subsubsection{Compressão de Imagens}
\par

O utilizador pretende igualmente ver as imagens com a máxima qualidade, mais qualidade significa um maior detalhe e por sequencia um ficheiro de maior tamanho. Existem atualmente vários softwares online e locais que reduzem o tamanho das imagens. Na conceção da página da Nidus é utilizado o website TinyPNG.com que analisa a imagem original e converte as cores em cores mais simples de o sistema armazenar, como por exemplo uma imagem com 24 bits de profundidade de cor pode ser convertido em uma similar com apenas 8 bits reduzindo o tamanho do ficheiro e impercetível para o olho humano num ecrã\cite{Hilles2019}. Alternativamente ao Tiny Png existem softwares, similares alguns de licença GNU/GPL, para comprimir imagens. Com o "Mass Image Compressor"\cite{Mass_Image_Compressor}(apenas um exemplo), é possível comprimir as imagens com a possibilidade de indicar a quantidade de compressão.
\par
Com a enorme quantidade e diversidade de monitores existentes, as páginas web necessitam de ser responsivas e apresentar a melhor imagem para o monitor em questão, isso normalmente traduz-se em várias versões similares da imagem alojadas no servidor. No caso dos microcontroladores e sistemas embebidos o espaço encontra-se limitado e deve-se arranjar uma solução. Uma solução possível é ao invés da utilização de imagens PNG, JPG ou outras, é a utilização de imagens em SVG, onde a imagem é representada por um ficheiro XML que descreve uma imagem bidimensional e utiliza na sua constituição modelos matemáticos para o cálculo das posições dos elementos. Com isto é possível manipular o XML em tempo real para alterar elementos ou remover, alterar cores, criar animações entre outras. Inclui a vantagem de como a imagem é representada por formulas matemáticas, é possível escalar a imagem sem perder qualidade pois a função matemática é ajustável. Num sistema embebido como o caso da Nidus é vantajoso a utilização de imagens em SVG para criação das animações. Atualmente as animações da página da Nidus são criadas com várias imagens PNG comprimidas e convertidas em base64 e são alternadas no HTML pelo JavaScript. Com a utilização de imagens SVG é possível ter apenas uma imagem alojada e manipular a imagem em tempo real através do JavaScript de uma forma mais suave para o utilizador, pois apenas a zona a alterar é alterada na imagem.
Á semelhança dos JPG e PNG o SVG também pode ser comprimido, para tal basta no XML da imagem remover os meta-dados  e utilização de funções matemáticas mais simples, não necessários para o browser apresentar a representação gráfica do mesmo, mas os softwares de edição adicionam para funcionalidades exclusivas do editor. Á semelhança dos ficheiros HTML após a remoção dos meta-dados o ficheiro pode ser minificado.

\subsubsection{Localização indoor}
\par
É possivel encontrar na comunidade científica vários estudos sobre a utilização de redes Wi-Fi e Bluetooth para sistemas de localização. Estes mesmos focam-se no cálculo das distâncias do equipamento para vários recetores no mesmo intervalo temporal, algumas destas soluções baseiam-senos valores de RSSI da transmissão e o valor definido como constante da potência de transmissão á distância de 1 metro, e estimar a sua distância aproximada de cada recetor, com essas aproximações é possivel através do algoritmo escolhido\cite{Wang2013}, obter a estimativa da localização do equipamento e a sua colocação num mapa.
A distância de um recetor para o emissor baseada no  valor de RSSI é expressada pela seguinte formula, onde dbm é a constante da potência de transmissão da beacon a 1 metro, n a constante do ambiente e o RSSI corresponde ao RSSI da transmissão:
\par
\begin{center}
  $d=10^(\frac{dbm-RSSI}{10 \times n})$
\end{center}

\par
Após a obtenção da distância para cada recetor é possível  aplicar um algoritmo para estimar a localização. Os mais referenciados e adotados são o centroid baseado no centro geométrico do polígono formado pelas interceções das circunferências criadas com o raio da distância calculada pela fórmula anteriormente apresentada, o método Three-border Positioning e o Least Square Estimation.
Como é possivel observar na figura \ref{centroid} utilizando o método do centroid, o centro geométrico corresponde á localização do equipamento com base nos recetores. A formula que representa o centro utilizando o centroid é expressada pela seguinte equação onde n representa o numero de recetores utilizados no cálculo.
\par
\begin{center}
$ (x,y)= (\frac{x_{1}+x_{2}+x_{3}+...+x_{n}}{n},\frac{y_{1}+y_{2}+y_{3}+...+y_{n}}{n})$
\end{center}

\begin{figure}[htb]
\centering
\includegraphics[width=0.95\textwidth]{images/centroid3.png}
\caption{Posição utilizando o método Centroid com 3 e 4 recetores}\label{centroid}
\end{figure}

Ao invés da utilização do método do centroid se for adotado o método Three-border Positioing, é criada a função definida por ramos composta pelas três equações da circunferência A, B e C com os respetivos centros em cada recetor e com o raio igual á distância calculada para esse mesmo recetor. Para calcular a posição estimada é calculado o resultado dessa mesma função de modo a encontrar o ponto x,y que representa a posição do equipamento.
\par 
Utilizando o método Least Sqare Estimation ou simplesmente LSE e á semelhança do Three-border Position\cite{Zhu2014} é criada a função de ramos das equações das circunferências dos vários recetores com o raio da distância calculada, mas pode igualmente como o centroid utilizar mais do que três recetores aumentando a precisão. \par Hua, Z., Hang, L., Yue, L., Hang, L., \& Kan, Z. (2014). Geometrical constrained least squares estimation in wireless location systems. 2014 4th IEEE International Conference on Network Infrastructure and Digital Content. apresenta os passos necessários calcular a Posicação X do equipamento através do método LSE. Em primeiro são criadas a função de ramos composta pelas equações das circunferências com centro nos recetores ([$x_{1}$,$y_{1}$],[$x_{2}$,$y_{2}$],[$x_{3}$,$y_{3}$],[$x_{4}$,$y_{4}$]) e o raio igual á distância calculada($d_{1}$,$d_{2}$,$d_{3}$,$d_{4}$).
\begin{center}
\[
  \begin{cases}
      (x_{1} -x)^2 + (y_{1}-y)^2 = {d_{1}}^2\\
      (x_{2} -x)^2 + (y_{2}-y)^2 = {d_{2}}^2\\
      (x_{3} -x)^2 + (y_{3}-y)^2 = {d_{3}}^2\\
      (x_{4} -x)^2 + (y_{4}-y)^2 = {d_{4}}^2\\
  \end{cases}
\]
\end{center}

\par Após a criação da função é subtraido o primeiro ramo aos restantes ramos e a função reduz o numero de ramos para n-1 onde n representa o numero de recetores a usar na função.
\begin{center}

\[
  \begin{cases}
     2(x_{2}-x_{1})x+2(y_{2}-y_{1})y={x_{2}}^2-{x_{1}}^2+{y_{2}}^2-{y_{1}}^2+{d_{2}}^2+{ d_{1}}^2\\
     2(x_{3}-x_{1})x+2(y_{3}-y_{1})y={x_{3}}^2-{x_{1}}^2+{y_{3}}^2-{y_{1}}^2+{d_{3}}^2+{ d_{1}}^2\\
     2(x_{4}-x_{1})x+2(y_{4}-y_{1})y={x_{4}}^2-{x_{1}}^2+{y_{4}}^2-{y_{1}}^2+{d_{4}}^2+{ d_{1}}^2\\
  \end{cases}
\]
\end{center}
\par A função pode ser representada pelo seu equivalente numa representação de matrizes por $2AX = b$ onde.

\begin{center}



$A=\begin{bmatrix}
x_{2}-x_{1} & y_{2}-y_{1}\\
x_{3}-x_{1} & y_{3}-y_{1}\\
x_{4}-x_{1} & y_{4}-y_{1}
\end{bmatrix}$

$B=\begin{bmatrix}
b_{1}\\
b_{2}\\
b_{3}
\end{bmatrix}=\begin{bmatrix}
{x_{2}}^2-{x_{1}}^2  + {y_{2}}^2-{y_{1}}^2 - {d_{2}}^2 + {d_{1}}^2 \\
{x_{3}}^2-{x_{1}}^2  + {y_{3}}^2-{y_{1}}^2 - {d_{3}}^2 + {d_{1}}^2 \\
{x_{4}}^2-{x_{1}}^2  + {y_{4}}^2-{y_{1}}^2 - {d_{4}}^2 + {d_{1}}^2 \\
\end{bmatrix} $

\end{center}

\par A posição estimada do equipamento representada no exemplo por X é definida por:



\par
\begin{center}
$ X= \frac{1}{2}(A^T A)^{-1} A^T b$
\end{center}

\par Os testes analisados demonstram\cite{Wang2013} , que o método LSE é o método que obtem os melhores resultados com os valores mais próximos do real. No teste apresentado em segundo lugar está o Three-border Position e por último o Centroid. Com algumas discrepâncias em algumas das amostragens.

\subsection{Produtos Similares}
\subsubsection{NB-Iot}
\par
Atualmente no mercado começam a surgir alguns produtos similares ao que se pretende desenvolver como é o caso dos sensores da Efento\cite{epoka}, que disponibiliza vários tipos de sensores que comunicam por NB-Iot. A Efento é uma empresa fundada em 2014 e é focada em desenvolvimento de equipamentos IOT. Atualmente desenvolveram versões com suporte para NB-Iot. Estes equipamentos tem a desvantagem de não ser compatível com o pacote de envio desenvolvido no portal Senslive e apenas permite o envio para o portal da Efento e não existe a possibilidade da utilização das sondas já comercializadas pela Captemp. Como vantagem á semelhança do equipamento a desenvolver é a utilização de um sistema com Log para quando não existe possibilidade de comunicação.
Devido ao desenvolvimento da tecnologia ainda existem poucas soluções em comercialização, estando as mesmas em desenvolvimento. A Captemp possui igualmente outro equipamento, completamente desenvolvido pela empresa, em desenvolvimento que tira partido do NB-Iot com o acréscimo em relação ao que se pretende desenvolver durante o estágio, a possibilidade de ter mais sensores, maior capacidade de Log interno, configuração por Bluetooth, GPS e um Display integrado como extra.
\subsubsection{Kea Tracker}
Após pesquisas online é possível encontrar algumas soluções de beacons que permitem o armazenamento interno de leituras para desenvolver um sistema de data-logger tais como a Beacon da Fujitsu, a FWM8BLZ02A-109069\cite{beacon1} , á semelhança da beacon da Ruuvi usa o mesmo chip o nRF52832 da Nordic Semiconductor, mas apresenta como vantagens a inclusão de um sistema de Logs interno com capacidade para aproximadamente 4080 leituras e a diversidade de sensores já incluídos. Como desvantagem em relação á Beacon da Ruuvi tem a inclusão de um sensor de temperatura ao invés de temperatura e humidade, não possui sensor de pressão atmosférica e não é open-source possuindo um firmware fechado. A vantagem de se desenvolver um produto desde a sua raiz é a possibilidade de ter o controlo total sobre a solução para posteriores melhoramentos e ter a solução a desempenhar apenas o que pretendemos.
\par
Outra solução existente no mercado é igualmente a solução da Blue Maestro que possui variadas versões de beacons. Á semelhança da Beacon da Fujitsu possuem igualmente sistema de Log. Contrariamente á FWM8BLZ02A-109069 é uma beacon que tem disponível em Open-Source uma API e um SDK para desenvolver as nossas aplicações. Comparada com a beacon da Ruuvi, a Ruuvi beacon é completamente open-source e não apenas a API para comunicação.
\par
Na tabela \ref{tabbeacons} são apresentadas as diferenças e semelhanças entre os três modelos analisados

\begin{table}[htb]
\caption{Comparação entre beacons \cite{specsrect}\cite{bluespecs}\cite{ruuvispecs}}\label{tabbeacons}
\begin{tabular}{|c|c|c|c|}\hline
& Ruuvi Tag& Fujitsu Beacon &Blue Maestro \\\hline
Processador& nRF52832& nRF52832 &? \\\hline
Memória&\begin{tabular}{@{}c@{}}512kB Flash \\ 64kB RAM\end{tabular} & 32K Não volátil &?\\\hline 
Protocolos&\begin{tabular}{@{}c@{}c@{}@{}c@{}} Bluetooth 5 \\ Wirepass \\ Mira OS\\QUUPA\\Others (2.4GHz)\end{tabular}&Bluetooth 4.1&BLE 4.2\\\hline 
\begin{tabular}{@{}c@{}}Potência de\\ Transmissão\end{tabular} &+4 dBm &\begin{tabular}{@{}c@{}}-16, -12, -8\\ -4, 0, +4 dBm\end{tabular} &-4, 0, +4 dBm \\\hline
Sensores& \begin{tabular}{@{}c@{}c@{}c@{}} Acelerometro\\ Temperatura\\ Humidade \\Pressão\end{tabular} &\begin{tabular}{c@{}c@{}} Acelerómetro\\ Temperatura\end{tabular}&\begin{tabular}{@{}c@{}c@{}} Temperatura\\ Humidade \\Pressão\end{tabular}\\\hline 
NFC & \checkmark&- &-\\\hline
Bateria &\begin{tabular}{@{}c@{}}CR2477\\ 1000mAH - Li/MnO2\end{tabular}&CR2450 &CR2032\\\hline
\begin{tabular}{@{}c@{}}Autonomia\\(espetável)\end{tabular}& ~10 Anos&1 Ano em Broadcast &\begin{tabular}{@{}c@{}}1 Ano em Broadcast\\2 Anos com Log\end{tabular}\\\hline
Data Logger &\begin{tabular}{@{}c@{}}-\\(a desenvolver)\end{tabular}&\checkmark &\checkmark\\\hline
Open Source & \checkmark&- &\checkmark ( API \& SDK )\\\hline
Informações & \begin{tabular}{@{}c@{}c@{}@{}} IP67 \\ 2 Botões\\2 Leds\\52mm \diameter\\\end{tabular}&\begin{tabular}{c@{}c@{}} Led\\ 40 x 31 x 12mm \end{tabular}&\begin{tabular}{@{}c@{}}24000 Registos\\33mm \diameter\end{tabular} \\\hline
\end{tabular} 
\end{table}
\par


\chapter{Trabalho Desenvolvido}
\section{Introdução}
Nesta secção é apresentado o trabalho e desenvolvimento dos projetos realizados durante o estágio e á semelhança do capítulo anterior cada projeto será desenvolvido num subcapitulo dedicado.

\section{Coletor de Dados - Nidus} 
\par



\section {NB-Iot \& Digi Xbee 3 }
\par
O desenvolvimento do projeto NB-Iot \& Digi Xbee 3  é composto por 4 fases, 3 das quais desenvolvidas durante este estágio. A fase não desenvolvida durante o estágio refere-se ao desenho e produção do hardware e a parte do software referente á leitura de sensores (comunicação entre hardware desenvolvido e software). As fases realizadas durante o estágio são a implementação do envio do pacote definido com os mecanismos de proteção e segurança, sincronismo dos tempos de leitura e envio e testes ao sistema.

\subsection {Envio de dados para o portal}

\par Como foi indicado no Capítulo \ref{nbiot} a estrutura de pacote a enviar é similar ao do outro produto desenvolvido pela captemp. Este pacote é enviado através de um pacote UDP para o portal que posteiormente confirma a receção na camada de aplicação. O tamanho máximo definido para este pacote é de 1000 bytes.
A estrutura criada pela captemp segue o formato apresentado na figura \ref{packet}.
\par No Primeiro cabeçalho é possivel obter os dados do equipamento que fez o envio tais como a data de envio, o IMEI, a versão do mesmo e o CRC do pacote para confirmar a integridade do mesmo. No restante do pacote são adicionados vários sub pacotes seguindo a estrutura apresentada na figura \ref {packet}, o primeiro byte indica o tipo de dados se é envio o valor de um sensor uma configuraçao do equipamento por exemplo a operadora, o byte seguinte fornece o número de bytes dos dados e posterioremente segundo o número de bytes é dado o valor no exemplo indicado da operadora pode ser a string "ALTICE".

 \begin{figure}[ht]
\centering
\includegraphics[width=0.95\textwidth]{images/packetnb.png}
\caption{Estrutura do Pacote NB-IOT}\label{packet}
\end{figure}

\subsection {Gestão de memória}

\par O principal motivo da desistencia da utilização deste equipamento como o equipamento principal da Captemp para o NB-iot é a incorreta gestão de memória do MicroPython. Segundo a documentação quando uma variável já não é acessivel pelo código esta é removida pelo \textit{Garbage Collector} mas o espaço de memória  ocupado fica disponivel mas não é mais usado pelo MicroPython e este aloca no final ao invés de procurar o primeiro espaço disponivel. No esquema apresentado na figura \ref{memo} é apresentado o comportamento da memória com a gestão nativa do MicroPython, com o decorrer do tempo a memória fica totalmente alocada não permitindo ao equipamento guardar novas leituras nem enviar para o portal.


 \begin{figure}[ht]
\centering
\includegraphics[width=0.95\textwidth]{images/memo.png}
\caption{Comportamento da gestão de Memória}\label{memo}
\end{figure}



\subsubsection {Diminuição da alocação de memória}

\par A modo de resolver a incorreta gestão de memória é necessário diminuir a alocação de novas variáveis. Para tal todo o código do programa visa a possuir todas as constantes e variáveis em variaveis globais as quais nunca são eliminadas e criadas apenas alteram o seu valor. Os dois pontos criticos identificados são o array circular com as leituras dos sensores e a criação do pacote de envio para o portal Cloud.
\par Na criação do pacote todos os cabeçallhos fixos são alocados em constantes globais que não vem o seu valor alterado não afetando a memória e os valores que podem ser alterados são guardados em buffers globais criados no inicio e tem o seu tamanho estático. Como o módulo utilizado não possui suporte para multrithread não existe nunhum problema de sincronismo ao utilizar as variáveis e os buffers globais. Ao recolher um dado de modo a enviar para o portal, como por exemplo a operadora, este tem de ser alocado num buffer que depois é enviado pela rede para o servidor. Este buffer de bytes com o tamanho fixo do máximo do pacote de envio é utilizado para fazer a contatenação dos vários campos antes de enviar. Caso se pretenda adicionar valores a enviar é consultada a ultima posiçao ocupada, encontrada numa váriavel separada e é alterado os bytes das posições seguintes para o bytes do valor não alocando espaço para continuar o array.
\par Supondo que o buffer tem já preenchidos 300 bytes dos 1000, ao pretender adicionar o pacote da operadora, é copiado para a posição 301 o byte correspondente áo DATA ID do operador, no byte seguinte é colocado o tamanho de bytes que a operadora ocupa ("ALTICE"= 6 Bytes), e nos seguintes 6 bytes é colocada a string.
\par No exemplo acima elucidado todas a variável DATA ID é estática e definida no inicio do código, o tamanho foi previamente guardado numa variavel auxiliar de tamanho fixo, e a operadora é soilicatado a funções nativas do MicroPtyhon  e guardado num buffer de tamanho fixo. Após a definição apenas são efetuadas copias de bytes entre variaveis e buffers não afetando a alocação de memória. No código seguinte é exempleficado  a operação anteriormente apresentada. Neste caso de modo a simplificar é apresentado um buffer do tamanho da operadora, no projeto foram criados buffers do tamanho 1,2,4,10,16,32 bytes consoante os valores mais comuns, no caso particular de um valor que possa ter por exemplo 20 bytes este é guardado no buffer de 32 e no momento da gravacao apenas são copiados os 20 primeiros bytes.

	
\begin{verbatim}
	c=bytearray(1000) # Packet Array
	clean=bytearray(1000) # Empty Packet Array 

	#BUFFERs

	cmd6=bytes(6)
	cmdID=bytes(1)
	cmdLEN=bytes(1)

	#DATA IDs
	c0=bytes([0x01])
	
	(...)	

	cmdID = c0 #  DATA ID
	cmdLEN = (6).to_bytes(1, 'little') # Data Size
	cmd6 = (oper).to_bytes(2, 'little') # Data
	byteschange(c, 300, 301, cmdID) # Copy to packet array 
	byteschange(c, 301, 302, cmdLEN)  # Copy to packet array 
	byteschange(c, 302, 308, cmd6)  # Copy to packet array 
 \end{verbatim}

\par Ao copiar valores de entre buffers e não alocando espaço para o array crescer é necessário um maior controlo nos tamanhos das variáveis de modo a não copiar valores de buffers vazios ou posições enexistentes. A versão do MicroPython disponibilizada pela Digi não incorpora a biblioteca que faz a gestão de arrays limitando não possibilitando a copia direta de arrays para outros arrays indicando apenas a posição inicial. Para tal a função byteschange definida previamente no código, simula essa operação onde apenas indicamos o buffer de destino, a posiçao inicial, a final e a origem da cópia. Esta função é responsável por verificar se os tamanhos são possiveis de copiar e copia posição a posição (byte a byte no caso apresentado) para as posiçoes entre os valores indicados. No fim do pacote ser enviado é possivel limpar o buffer chamando a mesma função indicando como origem da copia o buffer clean, um buffer constante do mesmo tamanho mas com os bytes todos vazios. Todos os buffers utilizados ao longo do projeto tem um semelhante em tamanho mas completamente vazio. Deste modo é possivel utilizar sempre os mesmos buffers e não existir a necessidade de alocar buffers ao longo do programa.

\subsubsection {Gestão da memória durante leituras}

\par À semelhança do buffer onde é criado o pacote enviado para o portal, as leituras são um dos pontos criticos referente á alocação de memória. Para tal á semenlhança do pacote de envio, é inicializado no inicio do programa, um array de tamanho fixo e com cada posição com um array do tamanho máximo de cada leitura. Ao adicionar uma nova leitura são colocados nos buffers intermédios todos os dados e são copiados para a posição seguinte á ultima posiçã ocupada. Caso a ultima posição ocupada corresponda à ultima do array, é adicionado sobre a primeira posição, criando assim o array circular.
\par Todas as operações para adicionar uma leitura ao array ou remover são efetuadas com a função byteschange de  copiando o buffer temporátio com a leitura ou o vazio respetivamente. Na figura \ref{circbuf} é apresentado o esquema do array circular com as leituras. Como array é definido inicialmente e o utilizador pode alterar as sondas enquanto o equipamento está em funcionamento é sempre alocado para cada leitura o máximo de sensores garantindo que caso o utilizador adicione um sensor não seja necessário expandir o tamanho da posiçao no array causando os problemas de memoria já identificados. Para eliminar a posição é decrementado a posição currente do array e por proteção de modo a posteriormente não acedermos a dados que possam la existir, como por exemplo numa leitura antiga com 6 sensores e nova com 3 sensores os ultimos 3 sensores da leitura antiga ainda estavam associados apesar de no cabeçalho da leitura indicar que eram apenas 3. Zerando o buffer garantimos que por algum lapso ocorra leituras erradas da zona do buffer a mesma nao irá afetar nenhuma leitura.


 \begin{figure}[ht]
\centering
\includegraphics[width=0.85\textwidth]{images/circbuf.png}
\caption{Array Circular com leituras}\label{circbuf}
\end{figure}

\subsection {Sincronismo de Leitura e Envio} 

\par À semelhança dos restantes produtos produzidos pela captemp existe sempre sincronismo de leituras entre os equipamentos. Para tal além de cada equipamento possuir um relógio interno (RTC), é necesário fazer o sincronismo na primeira leitura, ou seja caso esteja programado para fazer a leitura a cada 5 minutos, existe uma diferença caso o mesmo seja ligado ás 00:00 e fazer as leituras respetivamente às 00:05, 00:10,00:15,00:20 , mas caso seja ligado por exemplo às 00:12, faz leituras às 00:17,00:00:22 e por diante não existindo um sincronismo entre os vários equipamentos dos clientes.
\par Para tal o equipamento no momento inicial do arranque verifica qual o intervalo de leitura e de envio, e verifica caso fosse ligado pelas 00:00 qual seria a proxima leitura ao momento atual, neste momento o equipamento não entra em modo de poupança de energia durante os 5 minutos até á leitura e apenas o tempo restante até ao valor caso fosse ligado pelas 00:00, no momento da leitura o equipamento já entra em modo de poupança de energia durante o tempo definido (Ex:5 minutos). Este método é igualmente aplicado ao intervalo de envio. Aquando do envio é enviado na resposta proveniente do servidor o \textit{Timestamp} do servidor de modo a sincronizar o tempo de todos os equipamentos e configurar nos novos equipamentos que sejam ligados e ainda possuam o valor default do RTC, no caso do escolhido no projeto 1 de Janeiro de 2000.
\par Caso o equipamento possuir uma data inferior á estipulada de 1 de Janeiro de 2019, o mesmo não faz leituras, esperando pelo intervalo de envio para enviar um pacote apenas com os campos fixos e sem leituras de modo a obter a resposta com o \textit{Timestamp} do servidor para iniciar as leituras.


\subsection {Encriptação dos dados}

\par De modo a proteger os dados na rede todo o pacote é encriptado con recurso á biblioteca MicroPython-AES disponibilizada online !!!!!!!!!!!!!!! colocar link do github. !!!!!!!!!!!!!!!!!!!!!!!. De modo a exta biblioteca desenvolvida para o MicroPython Base e não para a versão disponibilizada nos módulos DIGI, é necessário reconstruir a biblioteca para nao utilizar a biblioteca dos arrays  indisponivel nesta versão e fazer a alteração para a função byteschange ou aquando da necessidade de acrescentar dados no array colocar posição por posição. 


\begin{verbatim}
...
for offset in range(0, len(data), block_size):
    block = data[offset:offset + block_size]
    block_func(block)
    data[offset:offset + block_size] = block # ERROR 
    #['array' object does not support item assignment]
...

...
for offset in range(0, len(data), block_size):
    block = data[offset:offset + block_size]
    block_func(block)
    # Solution ['array' object does not support item assignment]
    for i in range(block_size)
        data[offset+i]= block[i]
...
 \end{verbatim}

\section {Kea Tracker}

\par De modo a substituir um produto descontinuado e apresentar novas soluções aos clientes, irão ser utilizadas beacons da Ruuvi Tag para



\section {dot.Tracker}
\par
A pedido de um cliente, foi proposto o desenvolvimento de uma plataforma WEB para fazer a monitorização de pessoas e objetos em tempo real. O projeto passou por várias  etapas das quais destacam-se a análise dos requesitos do cliente, anáilise de tecnologias disponiveis, análise de soluções existentes já em comercialização, o desenvolvimento do portal Web, desenvolvimento do Back-End, e testes ao sistema. Apesar do projeto ser apenas desenvolvido por um elemento, mas devido á maior complexidade e duração do mesmo foi adotada a metodlogia SCRUM com entregas/apresentações ao cliente para obter o feed-back do trabalho desenvolvido e assim poder alterar alguns dos requesitos solicitados.
\par O Cliente indicou que devia ter atualizações dos mapas em tempo real, alertas enviados para o cliente WEB caso este esteja online e por email.
É assim possivel definir a tabela, apresentada na tabela \ref{tab1} com os requisitos da solução e a sua importância no desenvolvimento.

\begin{table}[htb]
\centering
\caption{Requesitos da Solução}\label{tab1}
\begin{tabular}{|p{3cm}|p{8cm}|p{2cm}|}\hline
Requesito&Descrição&Importância (1-10)\\\hline

Portal Cloud (Front-End)&Portal Cloud com mapas em tempo Real& 7\\\hline
Portal Cloud (Back-End) & API REST para integração com o Front-End e recolha dos dados para a localização &9\\\hline
Histórico de posições&Possibilidade de revisualizar no mapa o percurso entre datas&5\\\hline
Alertas Email&Alertas de Email (Exemplo: Entrada e Saída de zonas criadas no mapa)&6\\\hline
Alertas Web&Alertas Informativos no Mapa(Exemplo: Entrada e Saída de zonas criadas no mapa)&2\\\hline
\end{tabular} 
\end{table}

\par
De seguida são apresentados as funcionalidades e objetivos de cada requesito e escolhas selecionadas.

\subsection {Portal Cloud -  Front-End}

\par O Front-end da solução é desenvolvido com recurso á Framework Vue, tornando a solução numa solução Single-Page Aplication. A adoção da Framework é baseada na necessidade de possuir fluidez na navegação entre páginas e igualmente nos mapas em tempo real minimizando o delay.
\par A plataforma é capaz igualmente de suportar várias \textit{Companies}, significa isto que é possivel criar várias \textit{Companies} e temos os administradores e os utilizadores normais de cada \textit{Company} que apenas tem acesso ás suas definições e equipamentos. Possibilitando fornecer o projeto como uma solução Cloud a váriados clientes no mesmo Servidor, onde cada um apenas possui o acesso ao que pertence à sua \textit{Company}.
No final o utilizador da plataforma deve ser capaz de realizar as seguintes operações:
\par
\begin{itemize}
\item Login na Plataforma para visualizar os dados
\item Vizualizar um mapa com atualizações em tempo real
\item Editar o seu perfil
\item Visualizar a página numa língua á sua escolha
\item Utilizar a plataforma em vários equipamentos PC,Tablet,Smartphone,...
\item Gerir Utilizadores (Administradores)
\item Gerir Equipamentos (Administradores)
\item Gerir Mapas e Zonas (Administradores)
\item Gerir Alertas (Administradores)
\item Iniciar/ Finalizar Missões (Administradores)
\end{itemize}
\par


\subsection{ Portal Cloud - Back-end}

\par O Back-End é responsável por fornecer uma API REST ao Front-end para o mesmo obter os dados da base de dados. É igualmente responsável por obter os dados proveninentes dos gateways das beacons e calcular as suas posiçoes para apresentar no mapa. 
\par O primeiro problema a identificado no back-end é a necessidade de possuir atualizações em tempo real das posições. O Front-end não é capaz de calcular quando uma beacon comunica com o back-end apesar da mesma enviar o broadcast em tempos regulares, mas tanto a beacon  pode não estar ao alcance do gateway, como a mesma pode apenas estar ao alcance de menos de X(dependendo do algoritmo) gateways impossibilitando a utilização do algoritmo. 
\par Para tal além do serviço web é disponibilizado um servidor de WEB-Sockets para comunicações em tempo real entre o Back-end e o Front-End. Os WEB-Sockets é uma tecnologia que permite aos browsers mais recentes, ter um canal em tempo real com o back-end sem a necessidade de fazer um pedido HTTP, o que acarreta todo o processo do protocolo, como por exemplo o \textit{3-Way Handshake}. No momento da criação do websocket é criado uma ligação TCP a qual não é finalizada até ao fechar do websocket, o que elemina o sobrecarregamento da criação de pedidos e soluciona o problema da comunicação em tempo real para as atualizações dos mapas.
\par Outro problema e solucionar deparado na análise das tecnologias e soluções existentes é a falta de sincronismo da comunicação dos gateways. Supondo um exemplo com 5 gateways e 1 beacon. A beacon ao intervalo de tempo X1 comunica o pacote e apenas 4 gateways recebem o pacote e o enviam para o servidor através de um pedido HTTP. O servidor apesar de ter configurado 5 gateways não é capaz de prever se o 5º gateway irá comunicar a transmissão da beacon, o mesmo pode não estar ao alcance, pode não ter comunicação ao servidor, pode estar desligado ou pode haver algum problema na rede que atrase a chegada do pacote. Igualmente por variados motivos os 4 gateways que enviaram o pacote ao servidor não irão chegar todos em simultaneamente, criando o problema "Já chegaram todos os pacotes? Calculo com os que tenho, ou espero que chegue mais algum pacote?",
\par O primeiro passo para resolver o problema acima citado ao contrário das soluções mais tradicionais não irá ser utilizado um serviço WEB como o APACHE2 ou o NGINX, pois o mesmo não possui nenhum sincronismo entre pedidos e seria necessário armazenar momentaneamente todos os pacotes na base de dados e ter a tabela em constante escrita e leitura, não sendo o mais eficaz no cenário deste projeto. Para tal o serviço web irá ser responsabilidade do NodeJS(módulo express) que irá ter além do serviço WEB o servidor de WEB-Sockets, centralizando assim os dois serviçoes e possibilitando igualmente durante o processamento do pedido HTTP o envio de mensagens através do WEB-Socket. 
\par Na lista apresentada de seguida estão selecionados os pontos principais das funcionalidades do Back-End:

\par
\begin{itemize}
\item API REST para o Front-End (Login+ Dados)
\item API REST para o POST dos Gateways
\item Serviço WEB (express) para disponibilizar o Front-End 
\item Serviço WEB-Sockets 
\item Algoritmo de posicionamento
\item Envio de Alertas
\end{itemize}
\par



\par


\chapter{Testes e Avaliação}
\par Inserido num contexto empresarial , e tratando-se de projetos em desenvolvimento e de desenvolvimento contínuo, estes não foram concluidos com o concluir deste relatório e estágio. Todos os objetivos defindos no inicio do estágio foram concluidos com sucesso e com bons resultados como apresentado no capítulo \ref{cap4}. 
\par A realização de um estágio com projetos reais em ambiente empresarial permitiu uma melhor compreensão dos vários fatores que  não são possiveis replicar em ambiente académico, além da utilizaão de novas metodologias e técnologias.

\par Para trabalho futuro prevê-se a implementação do plugin Blockly na página da Nidus, algumas alterações simples de Design na página e a reformulação da ferramenta de compressão da página para a utilização do Brotli.
O projeto do Nb-Iot caso exista atualizações de \textit{Firmware} por parte da Digi que permitam a comunicação por Bluetooth, será implementada a configuração do equipamento através de uma aplicação móvel. No projeto dot.Tracker caso a necessidade de aumentar a precisão se venha a verificar, deve-se ponderar a migração para um sistema diferente de localização indoor sem recurso ao valor de RSSI tal como o \textit{Bluetooth} 5.1, em desenvolvimento, especialmente desenhado para localização em ambientes fechados. O \textit{Bluetooth} 5.1 com recurso a várias antenas por recetor, consegue calcular o ângulo entre o emissor e o recetor e com recurso a vários recetores delimitar a posição do emissor.

\par A utilização de \textit{Frameworks} que permitam o desenvolvimento de aplicações multiplataforma, tais como a utilizada neste estágio, o IONIC, é uma mais valia, pois utilizando o mesmo código fonte desenvolvido, permite além de manter o Design similar entre as plataformas, diminui o tempo de desenvolvimento quando são necessárias correções e atualizações e caso não seja necessário pelos requiesitos da aplicação, abstrai o desenvolvimento da programação nativa em cada ambiente, para \textit{Frameworks} de mais alto nível.

\par O estágio na Captemp foi uma experiência enriquecedora, onde foi possivel obter e consolidar conhecimento já adquiridos e novos, experiência e criadas novas relações quer a nível profissional quer pessoal.

\chapter{Conclusões e Trabalho Futuro}
\par Inserido num contexto empresarial , e tratando-se de projetos em desenvolvimento e de desenvolvimento contínuo, estes não foram concluidos com o concluir deste relatório e estágio. Todos os objetivos defindos no inicio do estágio foram concluidos com sucesso e com bons resultados como apresentado no capítulo \ref{cap4}. 
\par A realização de um estágio com projetos reais em ambiente empresarial permitiu uma melhor compreensão dos vários fatores que  não são possiveis replicar em ambiente académico, além da utilizaão de novas metodologias e técnologias.

\par Para trabalho futuro prevê-se a implementação do plugin Blockly na página da Nidus, algumas alterações simples de Design na página e a reformulação da ferramenta de compressão da página para a utilização do Brotli.
O projeto do Nb-Iot caso exista atualizações de \textit{Firmware} por parte da Digi que permitam a comunicação por Bluetooth, será implementada a configuração do equipamento através de uma aplicação móvel. No projeto dot.Tracker caso a necessidade de aumentar a precisão se venha a verificar, deve-se ponderar a migração para um sistema diferente de localização indoor sem recurso ao valor de RSSI tal como o \textit{Bluetooth} 5.1, em desenvolvimento, especialmente desenhado para localização em ambientes fechados. O \textit{Bluetooth} 5.1 com recurso a várias antenas por recetor, consegue calcular o ângulo entre o emissor e o recetor e com recurso a vários recetores delimitar a posição do emissor.

\par A utilização de \textit{Frameworks} que permitam o desenvolvimento de aplicações multiplataforma, tais como a utilizada neste estágio, o IONIC, é uma mais valia, pois utilizando o mesmo código fonte desenvolvido, permite além de manter o Design similar entre as plataformas, diminui o tempo de desenvolvimento quando são necessárias correções e atualizações e caso não seja necessário pelos requiesitos da aplicação, abstrai o desenvolvimento da programação nativa em cada ambiente, para \textit{Frameworks} de mais alto nível.

\par O estágio na Captemp foi uma experiência enriquecedora, onde foi possivel obter e consolidar conhecimento já adquiridos e novos, experiência e criadas novas relações quer a nível profissional quer pessoal.

% REFERENCES
% Edit the references.bib file to add your own references, that you can then
% \cite on your text.


\bibliographystyle{ieeetr}

\addcontentsline{toc}{chapter}{Bibliografia}
\bibliography{references}

%\clearpage %\cleardoublepage %for openright
\renewcommand{\thesection}{Apêndices \Roman{section}}

\begin{appendices}
\chapter{Exemplo de um SVG} \label{svg}
\begin{verbatim}

<?xml version="1.0" encoding="UTF-8" standalone="no"?>
<!-- Created with Inkscape (http://www.inkscape.org/) -->

<svg
   xmlns:dc="http://purl.org/dc/elements/1.1/"
   xmlns:cc="http://creativecommons.org/ns#"
   xmlns:rdf="http://www.w3.org/1999/02/22-rdf-syntax-ns#"
   xmlns:svg="http://www.w3.org/2000/svg"
   xmlns="http://www.w3.org/2000/svg"
   xmlns:sodipodi="http://sodipodi.sourceforge.net/DTD/sodipodi-0.dtd"
   xmlns:inkscape="http://www.inkscape.org/namespaces/inkscape"
   width="10mm"
   height="10mm"
   viewBox="0 0 10 10"
   version="1.1"
   id="svg8"
   inkscape:version="0.92.5 (2060ec1f9f, 2020-04-08)"
   sodipodi:docname="desenho1.svg">
  <defs
     id="defs2" />
  <sodipodi:namedview
     id="base"
     pagecolor="#ffffff"
     bordercolor="#666666"
     borderopacity="1.0"
     inkscape:pageopacity="0.0"
     inkscape:pageshadow="2"
     inkscape:zoom="15.839192"
     inkscape:cx="56.676533"
     inkscape:cy="22.492236"
     inkscape:document-units="mm"
     inkscape:current-layer="layer1"
     showgrid="false"
     inkscape:window-width="1920"
     inkscape:window-height="1017"
     inkscape:window-x="-8"
     inkscape:window-y="-8"
     inkscape:window-maximized="1" />
  <metadata
     id="metadata5">
    <rdf:RDF>
      <cc:Work
         rdf:about="">
        <dc:format>image/svg+xml</dc:format>
        <dc:type
           rdf:resource="http://purl.org/dc/dcmitype/StillImage" />
        <dc:title></dc:title>
      </cc:Work>
    </rdf:RDF>
  </metadata>
  <g
     inkscape:label="Layer 1"
     inkscape:groupmode="layer"
     id="layer1"
     transform="translate(0,-287)">
    <rect
       style="fill:#ff0000;stroke-width:0.26458332"
       id="rect10"
       width="9.1205721"
       height="4.660512"
       x="0.3674956"
       y="287.3783" />
    <circle
       style="fill:#00ffff;stroke-width:0.26458332"
       id="path12"
       cx="3.5747297"
       cy="293.07449"
       r="2.3887215" />
    <text
       xml:space="preserve"
       style="font-style:normal;font-weight:normal;font-size:1.2726717px;
line-height:1.25;font-family:sans-serif;letter-spacing:0px;word-spacing:0px;
fill:#000000;fill-opacity:1;stroke:none;stroke-width:0.03181679"
       x="8.6893291"
       y="242.43219"
       id="text18"
       transform="scale(0.82155341,1.2172063)"><tspan
         sodipodi:role="line"
         id="tspan16"
         x="8.6893291"
         y="242.43219"
         style="stroke-width:0.03181679">SVG </tspan></text>
  </g>
</svg>


\end{verbatim}
 \chapter{Exemplo de um SVG comprimido} \label{svggo}
\begin{verbatim}

<svg xmlns="http://www.w3.org/2000/svg" width="10mm" height="10mm" 
viewBox="0 0 10 10">
    <g transform="translate(0 -287)">
        <path fill="red" d="M.367 287.378h9.121v4.661H.367z" />
        <circle cx="3.575" cy="293.074" r="2.389" fill="#0ff" />
        <text style="line-height:1.25" x="8.689" y="242.432" 
transform="scale(.82155 1.2172)" font-weight="400" font-size="1.273" 
font-family="sans-serif" letter-spacing="0" word-spacing="0" 
stroke-width=".032">
            <tspan x="8.689" y="242.432">SVG</tspan>
        </text>
    </g>
</svg>


\end{verbatim}
\chapter{Fluxograma do processo de obtençao de Leituras/Log}\label{E}
 \begin{figure}[htb]

\centering
\includegraphics[width=0.60\textwidth]{images/flux2.png}
%\caption{Fluxograma do processo de Registo}
\par







\end{figure}


 \chapter{Fluxograma do processo de receção de pacotes} \label{flux1}
\input{C}
\chapter{Código da obtenção da posicção - LSE}\label{D}
\begin{lstlisting}[caption=Implementação do algoritmo LSE em Javascript,label={JSLSE},language=JavaScript2]
function getPosLSE(d1,d2,d3,d4,n1,n2,n3,n4,scale){
    if(scale==null){
        return -1;
    }

    /*Convert values from String to Float*/
    scale=parseFloat(scale);
    d1=parseFloat(d1);
    d2=parseFloat(d2);
    d3=parseFloat(d3);
    d4=parseFloat(d4);
    
    /*matriz A
        | a  d |
        | b  e |
        | c  f |
    */
    var a=(n2.x)-(n1.x);
    var b=(n3.x)-(n1.x);
    var c=(n4.x)-(n1.x);
    var d=(n2.y)-(n1.y);
    var e=(n3.y)-(n1.y);
    var f=(n4.y)-(n1.y);
    
    /*transposta de A
        | a b c |
        | d e f |
    
        At * a
        | aa bb |
        | cc dd |
        */
    
    var aa=(a*a)+(b*b)+(c*c);
    var bb=(a*d)+(b*e)+(c*f);
    var cc=bb; /*(d*a)+(e*b)+(f*c);*/
    var dd=(d*d)+(e*e)+(f*f);
    
    //determinante de At*A
    
    var det=(aa*dd)-(bb*cc);
    if(det!=0)
    {
        /*pode proceguir ha inversa
    
        inversa
            | aaa  bbb |
            | ccc  ddd |
    
        formula
            | dd/det  -bb/det | 
            | -cc/det  aa/det |
        */
        var aaa=dd/det;
        var bbb=-(bb/det);
        var ccc=-(cc/det);
        var ddd=(aa/det);
    
        // Multiplicar por 1/2
        var aaaa=aaa*0.5;
        var bbbb=bbb*0.5;
        var cccc=ccc*0.5;
        var dddd=ddd*0.5;	
    
        /* multiplicar pela transposta
            | aaaa bbbb |   *  | a b c |
            | cccc dddd |      | d e f |
        */

        var a5=(aaaa*a)+(bbbb*d);
        var b5=(aaaa*b)+(bbbb*e);
        var c5=(aaaa*c)+(bbbb*f);
        var d5=(cccc*a)+(dddd*d);
        var e5=(cccc*b)+(dddd*e);
        var f5=(cccc*c)+(dddd*f);
    
        /* definir b
            | b1 |
            | b2 |
            | b3 |
        */

        var d12=Math.pow((d1*scale),2);
        var b1=Math.pow((n2.x),2) - Math.pow((n1.x),2) + Math.pow((n2.y),2) - Math.pow((n1.y),2) - Math.pow((d2*scale),2) + d12;
        var b2=Math.pow((n3.x),2) - Math.pow((n1.x),2) + Math.pow((n3.y),2) - Math.pow((n1.y),2) - Math.pow((d3*scale),2) + d12;
        var b3=Math.pow((n4.x),2) - Math.pow((n1.x),2) + Math.pow((n4.y),2) - Math.pow((n1.y),2) - Math.pow((d4*scale),2) + d12;
    
    
        /* Multiplicar por b
            | a5 b5 c5 |       *        | b1 |
            | d5 e5 f5 |                | b2 |
                                        | b3 |
        */
        var resX=(a5*b1)+(b5*b2)+(c5*b3);
        var resY=(d5*b1)+(e5*b2)+(f5*b3);
        return {"x":resX,"y":resY};
    }
    return -1;
}
\end{lstlisting}



\end{appendices}

\end{document}
