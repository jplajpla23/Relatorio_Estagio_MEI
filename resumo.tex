% Resumo em Português

\vspace{1cm}
\noindent
\par Este relatório de estágio foi realizado no âmbito do estágio inserido no Mestrado em Engenharia Informática-Internet das Coisas da Escola Superior de Tecnologia de Tomar do Instituto Politécnico de Tomar, e tem como objetivo colocar em ambiente real os conhecimentos adquiridos no percurso académico.

\par Este estágio abarca o trabalho desenvolvido em 4 projectos da empresa Captemp, associados à área da internet das coisas (IoT) e da monitorização de ambientes e/ou objetos com recurso a diversas soluções. O primeiro projeto é referente ao equipamento "Nidus" já existente. Este equipamento, disponibilizado em vários modelos, permite a ligação de vários tipos de sensores e atuadores, centralizando assim o sistema de monitorização. O equipamento é também capaz de fazer a comunicação dos dados para um portal \textit{Cloud} onde é possível analisar os dados enviados pelos vários equipamentos. Este projeto foi necessário a análise do estado atual do projeto , com base nesta, dar continuidade ao desenvolviemnto do \textit{Front-end} WEB do equipamento. Em maior detalhe, o foco principal neste projecto consistiu no estudo e reestruturarão do software de \textit{Front-end} usado, tendo em conta os recursos limitados do equipamento em causa.
Para este fim, foram consideradas diversas estratégias, como: a utilização de técnicas de compressão de código, de forma a reduzir o espaço ocupado por páginas WEB; a adopção de métodos mais optimizados para a representação de imagens nos interfaces WEB, como imagens vectoriais, reduzindo o armazenamento e tempos de transferência; desenvolvimento de um sistema de internacionalização mais simplificado, com uma pegada de código mais reduzida.
\par Os restantes projectos desenvolvidos ao longo deste estágio consistiram no desenvolvimento de soluções de raiz a pedido os clientes, por soluções à medida, ou quer pela inovação e evolução dos produtos da empresa, e consistem em novos equipamentos para monitorização.

\par O segundo projeto é referente ao desenvolvimento de um equipamento IOT que tire partido das vantagens da nova tecnologia de comunicação o \textit{Narrowband} (NB-IOT). Neste equipamento é possível adicionar vários sensores já comercializados pela Captemp. O equipamento é capaz de realizar as leituras de todos os sensores conectados, realizar o armazenamento em registos para posterior envio para o portal \textit{Cloud}, para consulta futura ou gestão da alarmística correspondente. Este equipamento é igualmente capaz de realizar uma configuração bidirecional de modo a ser possível realizar a sua configuração através do portal \textit{Cloud}.

\par Paralelamente aos projetos anteriores, impulsionado pelo aparecimento de novas tecnologias tais como os \textit{Beacons Bluetooth Low Energy} e e pelo interesse demonstrado por diversos clientes em ter acesso a uma nova solução de monitorização portátil e simples, foi criado o projeto "Kea Tracker". Este projecto é composto por \textit{Beacons} e uma aplicação \textit{Mobile} para \textit{Smartphones}, responsável por periodicamente realizar a leitura dos sensores presentes nos \textit{Beacons}, o seu armazenamento e posterior envio para o portal \textit{Cloud}, à semelhança do projeto anterior. Neste projeto o \textit{Beacon} é igualmente capaz de registar a informação em Log interno para caso exista falha de comunicação com a aplicação \textit{Mobile}, posteriormente esses dados serem descarregados quando o \textit{Smartphone} estiver disponível.

\par O último projeto referente deste estágio, foi solicitado pelo cliente que pretende uma plataforma WEB para localizar pessoas em ambientes \textit{indoor}. Esta solução baseia-se no desenvolvimento de uma solução composta por localizadores, \textit{Gateways} e uma plataforma WEB onde seja possível realizar o \textit{Tracking} de pessoas e de objetos em ambientes \textit{indoor} num mapa e assim gerir os tempos de acesso a zonas definidas, criar alertas, ou simplesmente gerir o stock dos armazéns. Este projeto, à semelhança do projeto "Kea Tracker", usa como tecnologia de suporte \textit{Beacons Bluetooth Low Energy} (BLE) para fazer o \textit{tracking} em tempo real.




\bigskip

\textbf{Palavras chave:}
IOT, Monitorização, Compressão WEB, Compressão de Imagens, NB-IOT, Beacon, BLE, Indoor Tracking, Bluetooth Tracking