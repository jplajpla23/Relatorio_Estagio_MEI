\par Inserido num contexto empresarial, e tratando-se de projetos em desenvolvimento e de desenvolvimento contínuo, é natural que estes não tenham sido terminados com o concluir deste relatório e estágio. Não obstante deste facto, todos os objectivos definidos com sucesso e atingindo bons resultados, como apresentado no capítulo \ref{cap4}. 
\par A realização de um estágio com projetos reais em ambiente empresarial permitiu uma melhor compreensão dos vários conceitos abordados durante o mestrado, que não são fáceis de replicar em ambiente académico.

\par Para trabalho futuro prevê-se a implementação do plugin Blockly na página da Nidus, algumas alterações simples de \textit{design} na página e a reformulação da ferramenta de compressão da página para a utilização do Brotli.
No caso do projecto Nb-Iot e caso sejam disponibilizadas actualizações de \textit{Firmware} por parte do fabricante (Digi) que permitam a comunicação por Bluetooth, será implementada a configuração do equipamento através da aplicação móvel desnevolvida pela Captemp para a configuração do seu outro equipamento NB-IOT.
\par No projeto dot.Tracker, caso a necessidade de aumentar a precisão se venha a verificar, deve ser ponderara a migração para um sistema diferente de localização \textit{indoor} baseado apenas no valor de RSSI, tal como o \textit{Bluetooth} 5.1, em desenvolvimento, especialmente desenhado para localização em ambientes fechados. O \textit{Bluetooth} 5.1 com recurso a várias antenas por recetor, consegue calcular o ângulo entre o emissor e o recetor e com recurso a vários recetores delimitar a posição do emissor. Outra melhoria possível a implementar em versões futuras é a possibilidade de um \textit{gateway} também transmitir um pacote para os restantes \textit{gateways} ao alcance deste, tornando possível ao algoritmo calcular a distência entre os vários \textit{gateways}, ao invés da necessidade de configuração inicial dos \textit{gateways} no mapa por parte do utilizador. Deste modo o algoritmo é capaz de se adaptar dinamicamente às interferências que possam ocorrer no espaço pretendido.

\par A utilização de \textit{Frameworks} que permitam o desenvolvimento de aplicações multiplataforma, tais como a utilizada neste estágio, o IONIC, é uma mais valia, pois utilizando o mesmo código fonte desenvolvido, permite entre outras, manter o \textit{design} similar entre as diferentes plataformas, diminuir o tempo de desenvolvimento, em especial quando são necessárias correcções e actualizações.
Esta abordagem tem ainda a vantagem de abstrair o desenvolvimento da programação nativa de cada ambiente, através do uso de frameworks de alto nível.

\par O estágio na Captemp foi uma experiência enriquecedora, onde foi possível consolidar os conhecimentos já adquiridos ao longo da tese, assim como explorar e obter novos conhecimentos. Para além da componente técnica, tornou possível ganhar experiência em ambiente real e criar relações humanas, tanto a nível profissional como pessoal.
