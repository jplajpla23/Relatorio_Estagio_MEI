% Resumo em Português

\vspace{1cm}
\noindent
\par Este relatório de estágio foi realizado no âmbito do estágio inserido no Mestrado em Engenharia Informática-Internet das Coisas da Escola Superior de Tecnologia de Tomar do Instituto Politécnico de Tomar, e tem como objetivo colocar em ambiente real os conhecimentos adquiridos no percurso académico.
\par
O estágio tem inerente 4 projetos associados à área do IOT e da monitorização de ambientes e/ou objetos com recurso a diversas soluções. Um dos projetos, "Nidus" já existia e é necessário a análise do projeto para dar continuidade ao suporte do \textit{Front-end}. Neste projeto é necessário estudar e alterar os métodos de desenvolvimento da página referentes á compressão dos ficheiros para reduzir o espaço ocupado pelas páginas WEB, a adoção de um melhor método para utilização de imagens nas interfaces WEB em equipamentos com baixos recursos e o desenvolvimento de novas funcionalidades tais como um sistema de internalização de modo a disponibilizar o equipamento noutros idiomas. Os restantes projetos serão desenvolvidos de raiz durante o estágio quer por pedido de soluções à medida pelos clientes quer pela inovação/evolução dos produtos da empresa, e consistem em novos equipamentos para monitorização, nomeadamente monitorização ambiental e \textit{Tracking} de objetos.
\par O segundo projeto é referente ao desenvolvimento do de um equipamento IOT que tira partido da nova tecnologia de comunicação o \textit{NB-IOT}. Neste equipamento é possivel adicionar vários sensores, e o equipamento é capaz de realizar as leituras dos diversos sensores, o armazenamento em Log para posterior envio para um portal \textit{Cloud} para consulta futura ou criação de alertas.
\par Paralelamente ao projeto anterior e com o emergir de novas técnologias tais como as \textit{Beacons Bluetooth Low Energy} e pela requesição por diversos clientes de uma nova solução de monitorização portátil e simples, foi criado o projeto "Kea Tracker" composto com várias \textit{Beacons} e uma aplicação \textit{Mobile} responsável por ler os sensores presentes nas \textit{Beacons} e o envio para o portal \textit{Cloud}. Neste projeto a \textit{Beacon} deve igualmente possuir um sistema de Log interno para a falha de comunicação com a aplicação \textit{Mobile}.
\par O último projeto do estágio, solicitado pelo cliente, baseia-se no desenvolvimento de uma solução composta por Sensores e \textit{Gateways} e uma plataforma WEB que seja possível realizar o \textit{Tracking} de pessoas e de objetos em ambientes \textit{indoor} e assim gerir tempos de acesso criar alertas, ou simplesmente gerir o stock de armazéns. Este projeto usa como técnologia de suporte \textit{Beacons Bluetooth Low Energy} (BLE) para fazer o tracking em tempo real.




\bigskip

\textbf{Palavras chave:}
IOT, Monitorização, Compressão WEB,Compressão de Imagens,NB-IOT,Beacon,BLE, Indoor Tracking, Bluetooth Tracking
