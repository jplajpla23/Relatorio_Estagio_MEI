\par Neste capítulo são abordados os os testes e a avaliação do trabalho desenvolvido durante o estágio e apresentado no capítulo \ref{workcharp}. À semelhança dos capítulos anteriores, o capítulo está divido em secções, uma por cada projecto. Todos os projetos obtiveram resultados positivos em relação aos objetivos traçados no início do estágio.

\section{Nidus}

\par Uma parte integrante do trabalho de estágio consistiu em testar as alterações e novas funcionalidades desenvolvidas, nomeadamente em relação à compressão dos ficheiros e das imagens. 
\par A compressão das imagens e utilização dos SVG ao invés dos PNG e JPEG, como foi indicado no subcapítulo \ref{custom}, foi implementada com sucesso e abriu novas possibilidades  no sistema Nidus. Esta alteração permitiu também melhorar a experiência de utilização para o utilizador, através do aumento da fluidez nas animações e transições de páginas.
\par A compressão dos ficheiros com recurso à mudança do método de compressão, originalmente o GZIP, para Brotli obteve bons resultados, como indicado na tabela \ref{tabw2}, nos ficheiros atuais da Nidus e permitiu libertar espaço suficiente para alocar as restantes funcionalidades desenvolvidas no estágio tais como o sistema de internacionalização, ou a adaptação futura do plugin Blockly na gestão de eventos.

\section{Nb-Iot}

\par Os testes realizados no \textit{firmware} desenvolvido demonstraram a capacidade de este utilizar toda a memória disponível sem se verificarem casos de necessidade de alocação de memória sem que esta esteja disponível. Em comparação com o \textit{firmware} inicial, o qual não suportava algumas funcionalidades tais como o o envio das suas configurações para o portal \textit{Cloud} Senslive, permitindo apenas o envio das leituras dos sensores, com um limite de 25 registos Log, o \textit{firmware} desenvolvido suporta agora todos os requisitos identificados no projecto, por exemplo a comunicação bidireccional entre o portal. Nos testes realizados ao \textit{firmware} desenvolvido para o equipamento, foi obtido um limite de 170 logs utilizando o valor máximo de sensores estipulados (6). Este valor pode ser configurado, permitindo assim aumentar a capacidade de Log (em número de registos). Por exemplo, estabelecendo como 3 o valor máximo de sensores aumenta o valor máximo de leituras em Log para 340. Este valor poderá ser levado até 1020 leituras, no caso de ser apenas alocado o espaço necessário o qual pode ser configurado e caso este seja alterado a possibilidades de aumentar a capacidade de Log é possível.

\section{Kea Tracker}

\par O trabalho realizado neste projecto focou-se no desenvolvimento da aplicação móvel multi plataforma, uma vez que as restantes partes do sistema foram fornecidas. A aplicação multiplataforma com o mesmo nome do projeto "Kea Tracker", integrou com sucesso todas as funcionalidades já existentes com sucesso e a inclusão das novas funcionalidades  referentes ao \textit{download} dos Logs e ao sincronismo de leituras que permite ao utilizador final uma precisão garantida ao minuto.

\section{dot.tracker}

\par Os testes realizados à nova versão da plataforma dot.Tracker demonstraram que esta é 5 a 6 vezes mais rápida relativamente ao protótipo desenvolvido com recurso à \textit{Framework} Laravel, para além das melhorias de desempenho, a nova versão vem também corrigir os problemas identificados tais como o sincronismo dos pedidos. Os filtros desenvolvidos/ aplicados conseguem com sucesso atenuar as oscilações presentes nas comunicações, garantindo um movimento mais estável e mais suave. 
\par  A adoção da correção de altura demonstrou ser de grande importância no cálculo da posição estimada do equipamento. Durante estes testes foi possível identificar um problema na utilização do método atual para cálculo da distância visto que no caso de ser configurada uma altura na plataforma e o equipamento se mover no eixo Z correspondente à altura os resultados obtidos aumentam o erro consoante a diferença de altura real e a calculada.  Ainda referente à altura no caso de oscilações muito grandes  e em casos específicos o algoritmo descarta pacotes visto não terem sentido. Para melhor ilustrar o problema, imaginemos um teste desta funcionalidade realizado com um \textit{beacon} e alguns recetores. Configurando na plataforma a altura do \textit{beacon} a 10 M e a altura do recetor a 0 M, caso seja rececionado um pacote com o RSSI correspondente a uma altura  perto de 10 M, é possível calcular a distância a colocar no mapa, tal como na figura \ref{altura1}. Caso o \textit{beacon} seja movido para uma altura de, por exemplo, 0.5 metros, o algoritmo desenvolvido vai compensar a distância com a diferença de alturas e não é capaz de calcular o valor correcto. Isto acontece uma vez que, segundo o teorema de Pitágoras, não é possível obter um triângulo retângulo com um cateto de 10 M e a hipotenusa de 0.5 M. Nestes casos a plataforma não calcula  a sua posição no mapa.
\par A respeito da precisão do sistema em geral, é possível obter uma precisão média entre 1 a 2 metros  consoantes os ambientes, o seu conteúdo ou simplesmente às interferências existentes. De modo a aumentar a  precisão em ambientes mais hostis para localização o aumento do número de recetores e menor distância entre eles permite melhores resultados.