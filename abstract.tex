% Abstract in English

\vspace{1cm}
\noindent
\textbf{} 

\par This traineeship report was made in the scope of the traineeship inserted in the master's degree in Computer Engineering -Internet of Things at the School of Technology of Tomar from the Polytechnic Institute of Tomar, and has a goal place in real environment the knowledge acquired in the academic path.


\par The traineeship has inherent 4 associated projects at the area of IOT and the monitoring of environments and/or objects using different solutions. The first project, referring to the existing "Nidus". This device, available in several models, allows the connection of several types of sensors and actuators and centralizes the monitoring system. This equipment is equally capable of send the data to a Cloud portal where it is possible to analyze the data sent by the various devices. In this project it is necessary to analyze the current status of the project and continue the support of the WEB Front-end of the equipment. In this project, it is necessary study and change the page development methods related to file compression to reduce the space occupied by the WEB pages in equipments with low computational resources and storage, the adoption of a better method for using images in the WEB interfaces in equipment with limited storage and the development of new features such as an internationalization system in order to make the equipment available in other languages, always taking into account, the low available storage. The rest of the projects will be developed from scratch during the traineeship or by customers requesting custom solutions, or by the innovation and evolution of the company's products, and consist in new equipments for monitoring.


\par The second project is relative of the development of an IOT device that takes advantage of the new communication technology, the Narrowband (NB-IOT). In this equipment it will be possible to add several sensors already sold by Captemp. The equipment is capable of performing the readings of all the connected sensors, making the storage in Log for later sending to the Cloud portal for future preview or management of the corresponding alarms. This equipment is also capable of performing a bi-directional configuration so that it is possible to carry out its configuration through the Cloud portal.


\par In parallel to previous projects and with the emergence of new technologies such as Bluetooth Low Energy Beacons and the suggestion of several customers for a new portable and simple monitoring solution, the "Kea Tracker" project was created, composed of Beacons and a Mobile application for Smartphones, responsible for periodically reading the sensors present in the Beacons, storing it and sending them to the Cloud portal, at similarity to the previous project. In this project Beacon is also able to register in the internal log in case there is a communication failure with the Mobile application, after which these data are downloaded when the Smartphone is available.


\par The last project of the traineeship, was requested by the client who wants a WEB platform to locate people in indoor environments. This solution is based on the development of a solution composed of locators, Gateways and a WEB platform where it is possible to perform the Tracking of people and objects in indoor environments on a map and manage access times in defined zones, create alerts, or simply manage warehouse stock. This project, like the "Kea Tracker" project, uses as support technology Bluetooth Low Energy Beacons (BLE) to do real-time tracking.



\bigskip

\textbf{Key words:} 
IOT, Monitoring, WEB Compression, Image Compression, NB-IOT, Beacon, BLE, Indoor Tracking, Bluetooth Tracking